%%%%%%%%%%%%
%% Please rename this main.tex file and the output PDF to
%% [lastname_firstname_graduationyear]
%% before submission.
%%%%%%%%%%%%

\documentclass[12pt]{caltech_thesis}
\usepackage[hyphens]{url}
\usepackage{lipsum}
\usepackage{graphicx}
\usepackage{float}
\usepackage{tabularx}
\usepackage{subcaption}
\usepackage{placeins}

\usepackage[acronym]{glossaries-extra}
\makeglossaries
% acronyms.tex
% Abbreviations used in the thesis (glossaries package)
% Usage:
%   \usepackage[acronym]{glossaries}
%   \makeglossaries
%   % acronyms.tex
% Abbreviations used in the thesis (glossaries package)
% Usage:
%   \usepackage[acronym]{glossaries}
%   \makeglossaries
%   % acronyms.tex
% Abbreviations used in the thesis (glossaries package)
% Usage:
%   \usepackage[acronym]{glossaries}
%   \makeglossaries
%   \input{acronyms}
%   % later in the doc
%   \printglossary[type=\acronymtype]
%
% Call with \gls{<key>} in text to print the short form and define on first use.
% -----------------------------------------------------------------------------
% --- Experiments & physics context
\newacronym[sort=01-01]{sm}{SM}{Standard Model}
\newacronym[sort=01-02]{hep}{HEP}{High-Energy Physics}
\newacronym[sort=01-03]{lhc}{LHC}{Large Hadron Collider}
\newacronym[sort=01-04]{cms}{CMS}{Compact Muon Solenoid}

% --- 02 CMS subsystems & trigger
\newacronym[sort=02-01]{ecal}{ECAL}{Electromagnetic Calorimeter}
\newacronym[sort=02-02]{hcal}{HCAL}{Hadron Calorimeter}
\newacronym[sort=02-03]{hlt}{HLT}{High-Level Trigger}
\newacronym[sort=02-04]{lone}{L1}{Level-1 Trigger}

% --- 03 Machine learning
\newacronym[sort=03-01]{ml}{ML}{Machine Learning}
\newacronym[sort=03-02]{lstm}{LSTM}{Long Short-Term Memory}
\newacronym[sort=03-03]{relu}{ReLU}{Rectified Linear Unit}

% --- 04 Adversarial ML
\newacronym[sort=04-01]{fgsm}{FGSM}{Fast Gradient Sign Method}
\newacronym[sort=04-02]{pgd}{PGD}{Projected Gradient Descent}
\newacronym[sort=04-03]{intprob}{PIP}{Probabilistic Integer Perturbation}
\newacronym[sort=04-03]{pip-pgd}{PIP-PGD}{Probabilistic Integer-Perturbed Projected Gradient Descent}

% --- 05 Metrics
\newacronym[sort=05-01]{roc}{ROC}{Receiver Operating Characteristic}
\newacronym[sort=05-02]{auc}{AUC}{Area Under the Curve}
\newacronym[sort=05-03]{jsd}{JSD}{Jensen--Shannon Distance}
\newacronym[sort=05-04]{kl}{KL}{Kullback--Leibler (divergence)}

% --- 06 PF reconstruction & inputs
\newacronym[sort=06-01]{pf}{PF}{Particle-Flow}
\newacronym[sort=06-02]{cpf}{CPF}{Charged Particle-Flow}
\newacronym[sort=06-03]{npf}{NPF}{Neutral Particle-Flow}
\newacronym[sort=06-04]{sv}{SV}{Secondary Vertex}
\newacronym[sort=06-05]{pv}{PV}{Primary Vertex}
\newacronym[sort=06-06]{puppi}{PUPPI}{PileUp Per Particle Identification}
\newacronym[sort=06-07]{csv}{CSV}{Combined Secondary Vertex}

% --- 07 Output discriminators
\newacronym[sort=07-01]{bvsl}{BvsL}{Bottom-versus-Light discriminator}
\newacronym[sort=07-02]{cvsb}{CvsB}{Charm-versus-Bottom discriminator}
\newacronym[sort=07-03]{cvsl}{CvsL}{Charm-versus-Light discriminator}

% --- 08 Confusion-matrix terms
\newacronym[sort=08-01]{tp}{TP}{True Positive}
\newacronym[sort=08-02]{tn}{TN}{True Negative}
\newacronym[sort=08-03]{fp}{FP}{False Positive}
\newacronym[sort=08-04]{fn}{FN}{False Negative}
\newacronym[sort=08-05]{tpr}{TPR}{True Positive Rate}
\newacronym[sort=08-06]{fpr}{FPR}{False Positive Rate}

%   % later in the doc
%   \printglossary[type=\acronymtype]
%
% Call with \gls{<key>} in text to print the short form and define on first use.
% -----------------------------------------------------------------------------
% --- Experiments & physics context
\newacronym[sort=01-01]{sm}{SM}{Standard Model}
\newacronym[sort=01-02]{hep}{HEP}{High-Energy Physics}
\newacronym[sort=01-03]{lhc}{LHC}{Large Hadron Collider}
\newacronym[sort=01-04]{cms}{CMS}{Compact Muon Solenoid}

% --- 02 CMS subsystems & trigger
\newacronym[sort=02-01]{ecal}{ECAL}{Electromagnetic Calorimeter}
\newacronym[sort=02-02]{hcal}{HCAL}{Hadron Calorimeter}
\newacronym[sort=02-03]{hlt}{HLT}{High-Level Trigger}
\newacronym[sort=02-04]{lone}{L1}{Level-1 Trigger}

% --- 03 Machine learning
\newacronym[sort=03-01]{ml}{ML}{Machine Learning}
\newacronym[sort=03-02]{lstm}{LSTM}{Long Short-Term Memory}
\newacronym[sort=03-03]{relu}{ReLU}{Rectified Linear Unit}

% --- 04 Adversarial ML
\newacronym[sort=04-01]{fgsm}{FGSM}{Fast Gradient Sign Method}
\newacronym[sort=04-02]{pgd}{PGD}{Projected Gradient Descent}
\newacronym[sort=04-03]{intprob}{PIP}{Probabilistic Integer Perturbation}
\newacronym[sort=04-03]{pip-pgd}{PIP-PGD}{Probabilistic Integer-Perturbed Projected Gradient Descent}

% --- 05 Metrics
\newacronym[sort=05-01]{roc}{ROC}{Receiver Operating Characteristic}
\newacronym[sort=05-02]{auc}{AUC}{Area Under the Curve}
\newacronym[sort=05-03]{jsd}{JSD}{Jensen--Shannon Distance}
\newacronym[sort=05-04]{kl}{KL}{Kullback--Leibler (divergence)}

% --- 06 PF reconstruction & inputs
\newacronym[sort=06-01]{pf}{PF}{Particle-Flow}
\newacronym[sort=06-02]{cpf}{CPF}{Charged Particle-Flow}
\newacronym[sort=06-03]{npf}{NPF}{Neutral Particle-Flow}
\newacronym[sort=06-04]{sv}{SV}{Secondary Vertex}
\newacronym[sort=06-05]{pv}{PV}{Primary Vertex}
\newacronym[sort=06-06]{puppi}{PUPPI}{PileUp Per Particle Identification}
\newacronym[sort=06-07]{csv}{CSV}{Combined Secondary Vertex}

% --- 07 Output discriminators
\newacronym[sort=07-01]{bvsl}{BvsL}{Bottom-versus-Light discriminator}
\newacronym[sort=07-02]{cvsb}{CvsB}{Charm-versus-Bottom discriminator}
\newacronym[sort=07-03]{cvsl}{CvsL}{Charm-versus-Light discriminator}

% --- 08 Confusion-matrix terms
\newacronym[sort=08-01]{tp}{TP}{True Positive}
\newacronym[sort=08-02]{tn}{TN}{True Negative}
\newacronym[sort=08-03]{fp}{FP}{False Positive}
\newacronym[sort=08-04]{fn}{FN}{False Negative}
\newacronym[sort=08-05]{tpr}{TPR}{True Positive Rate}
\newacronym[sort=08-06]{fpr}{FPR}{False Positive Rate}

%   % later in the doc
%   \printglossary[type=\acronymtype]
%
% Call with \gls{<key>} in text to print the short form and define on first use.
% -----------------------------------------------------------------------------
% --- Experiments & physics context
\newacronym[sort=01-01]{sm}{SM}{Standard Model}
\newacronym[sort=01-02]{hep}{HEP}{High-Energy Physics}
\newacronym[sort=01-03]{lhc}{LHC}{Large Hadron Collider}
\newacronym[sort=01-04]{cms}{CMS}{Compact Muon Solenoid}

% --- 02 CMS subsystems & trigger
\newacronym[sort=02-01]{ecal}{ECAL}{Electromagnetic Calorimeter}
\newacronym[sort=02-02]{hcal}{HCAL}{Hadron Calorimeter}
\newacronym[sort=02-03]{hlt}{HLT}{High-Level Trigger}
\newacronym[sort=02-04]{lone}{L1}{Level-1 Trigger}

% --- 03 Machine learning
\newacronym[sort=03-01]{ml}{ML}{Machine Learning}
\newacronym[sort=03-02]{lstm}{LSTM}{Long Short-Term Memory}
\newacronym[sort=03-03]{relu}{ReLU}{Rectified Linear Unit}

% --- 04 Adversarial ML
\newacronym[sort=04-01]{fgsm}{FGSM}{Fast Gradient Sign Method}
\newacronym[sort=04-02]{pgd}{PGD}{Projected Gradient Descent}
\newacronym[sort=04-03]{intprob}{PIP}{Probabilistic Integer Perturbation}
\newacronym[sort=04-03]{pip-pgd}{PIP-PGD}{Probabilistic Integer-Perturbed Projected Gradient Descent}

% --- 05 Metrics
\newacronym[sort=05-01]{roc}{ROC}{Receiver Operating Characteristic}
\newacronym[sort=05-02]{auc}{AUC}{Area Under the Curve}
\newacronym[sort=05-03]{jsd}{JSD}{Jensen--Shannon Distance}
\newacronym[sort=05-04]{kl}{KL}{Kullback--Leibler (divergence)}

% --- 06 PF reconstruction & inputs
\newacronym[sort=06-01]{pf}{PF}{Particle-Flow}
\newacronym[sort=06-02]{cpf}{CPF}{Charged Particle-Flow}
\newacronym[sort=06-03]{npf}{NPF}{Neutral Particle-Flow}
\newacronym[sort=06-04]{sv}{SV}{Secondary Vertex}
\newacronym[sort=06-05]{pv}{PV}{Primary Vertex}
\newacronym[sort=06-06]{puppi}{PUPPI}{PileUp Per Particle Identification}
\newacronym[sort=06-07]{csv}{CSV}{Combined Secondary Vertex}

% --- 07 Output discriminators
\newacronym[sort=07-01]{bvsl}{BvsL}{Bottom-versus-Light discriminator}
\newacronym[sort=07-02]{cvsb}{CvsB}{Charm-versus-Bottom discriminator}
\newacronym[sort=07-03]{cvsl}{CvsL}{Charm-versus-Light discriminator}

% --- 08 Confusion-matrix terms
\newacronym[sort=08-01]{tp}{TP}{True Positive}
\newacronym[sort=08-02]{tn}{TN}{True Negative}
\newacronym[sort=08-03]{fp}{FP}{False Positive}
\newacronym[sort=08-04]{fn}{FN}{False Negative}
\newacronym[sort=08-05]{tpr}{TPR}{True Positive Rate}
\newacronym[sort=08-06]{fpr}{FPR}{False Positive Rate}
 % <-- put acronyms.tex next to your main .tex

% Customize glossary spacing
\setlength{\glsdescwidth}{0.6\textwidth}
\renewcommand{\glsgroupskip}{2\baselineskip}
\renewcommand{\glspostdescription}{}
\renewcommand{\glsnamefont}[1]{\textbf{#1}}


\usepackage{todonotes}

%% Tentative: newtx for better-looking Times
\usepackage[utf8]{inputenc}
\usepackage[T1]{fontenc}
\usepackage{newtxtext,newtxmath}
\usepackage{xcolor}
\usepackage{colortbl}
\usepackage{microtype}
% Must use biblatex to produce the Published Contents and Contributions, per-chapter bibliography (if desired), etc.
\usepackage[backend=biber,style=numeric,sorting=none]{biblatex}

% Name of your .bib file(s)
\addbibresource{example.bib}
\graphicspath{ {./media/} }


\begin{document}



% Do remember to remove the square bracket!
\title{\textbf{A probabilistic approach for adversarial perturbation of integer-constraint features for Jet-Flavour Tagging Algorithms in the CMS Experiment}}
\author{Martin Waletzko}

\group{Physics Institute III A}
\faculty{Faculty of Mathematics, Computer Science and Natural Science}      % Faculty name
\university{RWTH Aachen University}    % Institution name
%\unilogo{caltech.png}                                 % Institution logo
\copyyear{2025}  % Year (of graduation) on diploma
\copymonth{August}

\supervisor{Prof.\ Dr.\ rer.\ nat.\ Alexander Schmidt \\ Physics Institute III A}
\secondexaminer{Prof.\ Dr.\ rer.\ nat.\ Johannes Erdmann \\ Physics Institute III A}


\maketitle

% Abstract
\begin{abstract}

Adversarial attacks pose a significant challenge to the robustness of deep-learning models used in high-energy physics analyses. This thesis investigates the vulnerability of jet-flavour tagging algorithms within the Compact Muon Solenoid experiment to perturbations of specifically discrete input features. In addition to standard gradient-based methods such as Projected Gradient Descent (PGD), PIP is introduced – a probabilistic integer attack designed for discrete-valued physics inputs. The attack leverages instantaneous gradient information to assign feature-specific flip probabilities, enabling efficient perturbations without continuous relaxations. A combined PIP-PGD attack is further studied to target both feature domains simultaneously. Evaluation on DeepJet demonstrates that PIP can induce competitive degradation relative to PGD, with complementary effects when applied in conjunction. Adversarial training experiments show improved robustness to the targeted attack types, though with limited generalisation across unseen perturbations. These results highlight the importance of accounting for discrete-feature vulnerabilities in the development of resilient high energy physics machine learning models.

\vfill
\noindent\fbox{\parbox{0.95\textwidth}{
\textbf{Disclaimer}\\
This thesis acknowledges the use of various artificial intelligence technologies, including but not limited to language models and coding assistants. These tools were exclusively employed for the enhancement of the clarity and comprehensibility of the text. The final content, including all conclusions and interpretations, is the sole responsibility of the author.
}}

\end{abstract}

\begin{acknowledgements}

I would like to take this opportunity to express my sincere gratitude to all those who supported and assisted me throughout the preparation of this thesis.

My deepest thanks go to Prof. Dr. Alexander Schmidt for providing me with the opportunity to prepare this thesis under his supervision, and for his trust while I was new to deep learning. His guidance and encouragement made it possible to pursue this work on adversarial robustness and jet-flavour tagging to completion.

I am also grateful to Prof. Dr. Johannes Erdmann for kindly agreeing to serve as the second examiner of this thesis and for his constructive feedback.

Moreover, I would like to thank Ulrich Willemsen and Alexander Jung for their assistance, insightful discussions, and the supportive working environment. In particular, I am thankful for their patience, valuable ideas, and technical support.

Finally, I would like to thank my family and friends for their encouragement and careful proofreading

\end{acknowledgements}

\tableofcontents
\listoffigures
\listoftables
\glsaddall[types=\acronymtype]
\clearpage
{%
  \let\cleardoublepage\clearpage
  \printglossary[type=\acronymtype,nonumberlist,style=long]%
}
%\printglossary[type=\acronymtype,nonumberlist]

\mainmatter

% Introduction
\chapter{Introduction}

Modern particle physics is becoming increasingly dependent on Machine Learning (ML) for essential analytical tasks, ranging from event reconstruction to particle identification \cite{Stein2022}. One notable
application of this technology is jet flavour tagging, where algorithms aim to distinguish jets that originate
from different quark flavours. The identification of bottom-quark jets plays a central role in many precision measurements and searches for new physics at the Large Hadron Collider (LHC) \cite{PhysRevLett.121.121801}. However, the robustness of these models is critical. In the broader ML community, adversarial attacks — small, deliberately designed perturbations that can mislead a model — have been widely studied, particularly in computer vision \cite{goodfellow2015explainingharnessingadversarialexamples}. Their potential impact in particle physics remains under-explored though, especially in scenarios involving discrete-valued input features.

Jet-tagging algorithms such as DeepJet \cite{Bols_2020} process hundreds of features, including continuous detector observables and discrete quantities (e.g. counts and category identifiers). Most adversarial studies affect only continuous inputs, overlooking the unique behaviour and vulnerabilities of models when discrete inputs are perturbed. This gap offers an opportunity to probe robustness in a more comprehensive way.

This thesis addresses this challenge by introducing Probabilistic Integer Perturbation (PIP), an adversarial attack tailored to discrete features. PIP uses gradient information in a one-step heuristic to assign feature-specific probabilities for discrete changes, ensuring that all perturbations remain physically meaningful. This method is furthermore combined with the Projected Gradient Descent (PGD) attack \cite{madry2019deeplearningmodelsresistant}, targeting both continuous and discrete domains simultaneously and providing a broader evaluation of model vulnerabilities.

Empirically, this study uses the DeepJet architecture \cite{Bols_2020} trained on approximately ten million simulated jets across all major flavour categories. In addition to the susceptibility for perturbation of continuous values, the results show that DeepJet is vulnerable to degradation through discrete perturbations too. The combined Probabilistic Integer-Perturbed Projected Gradient Descent (PIP-PGD) attack amplifies this effect. Adversarial training with PIP-PGD attacks is presented as a solution, offering robustness in the respective regimes.

Although discrete perturbations do not directly correspond to realistic detector effects, they serve as a diagnostic tool to reveal model dependencies on discretised inputs. Insights that remain hidden in continuous-only studies are revealed through the use of these perturbations. The work presented here contributes to the development of a more complete understanding of adversarial robustness in High Energy Physics (HEP) machine learning and outlines approaches for developing models that are both accurate and resilient in demanding scientific applications.

\newpage
The thesis is structured as follows: Chapters 2–3 provide the background and baseline, with Chapter 2 reviewing the Standard Model, the LHC, and the CMS detector, and Chapter 3 surveying ML in HEP and adversarial ML while introducing the DeepJet tagger used throughout. Chapter 4 formalises the dataset, evaluation metrics, and threat model, and presents the first methodological contribution: Probabilistic Integer Perturbation (PIP), a gradient-guided attack for discrete inputs that preserves integer constraints and physical bounds, together with the joint PIP–PGD attack that perturbs discrete and continuous features in conjunction. Chapter 5 constitutes the main experimental contribution: a systematic robustness study on DeepJet quantifying attack severity (via Jensen–Shannon distance), classifier performance (ROC/AUC), iteration depth, and PIP sharpness—and a defence via adversarial training, demonstrating that training with PIP–PGD achieves the most balanced cross-robustness while maintaining strong nominal performance; additionally, we examine transferability and cross-robustness across attack types. Chapter 6 summarises the findings, discusses limitations, and outlines directions for robust, discrete-aware jet tagging in future CMS analyses.

Together, these results argue for discrete-aware robustness assessments in HEP ML and provide practical methods to develop taggers that are not only accurate but also resilient in demanding scientific applications.


% Chapter 1
\chapter{The Standard Model and CMS}


\section{The Standard Model of Particle Physics}

The Standard Model (SM) provides a comprehensive framework that connects the known fundamental particles to three of the four fundamental forces: the strong force, the weak force, and the electromagnetic force. It organizes the elementary particles—specifically, the matter particles (quarks and leptons) and the force-carrying particles—under a unifying principle of symmetry. This symmetry ensures that the laws of physics remain consistent even as particles interact through these forces \cite{pich2012standardmodelelectroweakinteractions}.

In the SM, the strong force, which holds quarks together to form protons and neutrons, is carried by particles called gluons. The weak force, responsible for processes like radioactive decay, is mediated by particles known as the W and Z bosons. Meanwhile, the electromagnetic force, which controls how charged particles interact, is carried by the photon \cite{pich2012standardmodelelectroweakinteractions}. These forces are tied together through the SM’s symmetry, which helps explain why particles behave the way they do in nature.

The matter particles in the SM are divided into quarks and leptons. Quarks are the building blocks of protons and neutrons, while leptons include familiar particles like electrons and their neutral partners, neutrinos. These particles are arranged into three generations, each containing two quarks and two leptons (see Figure \ref{fig:standard_model}). The first generation includes the lightest and most stable particles, such as the up and down quarks (which make up protons and neutrons) and the electron and its neutrino. The second and third generations contain heavier, less stable particles that quickly decay into first-generation particles. This repeating family structure explains why protons and neutrons are made of first-generation quarks, while heavier particles are short-lived \cite{cush:standard-model}.

\begin{figure}[h]
    \centering
    \includegraphics[width=0.75\linewidth]{media/1024px-Standard_Model_of_Elementary_Particles.svg.png}
    \caption{Standard model of elementary particles: the 12 fundamental fermions and 5 fundamental bosons \cite{cush:standard-model}.}
    \label{fig:standard_model}
\end{figure}

A key aspect of the SM is how it accounts for the masses of certain particles. While the symmetry of the SM keeps the photon massless, the W and Z bosons, which are involved in the weak force, need to have mass to match experimental observations. This happens through a process where the symmetry is hidden in the everyday world but still governs the underlying physics, known as spontaneous symmetry breaking \cite{pich2012standardmodelelectroweakinteractions}. The Higgs field, a field that fills all of space and was proposed in the 1960s by physicists like Englert, Brout, and Higgs, makes this possible. The Higgs field gives mass to the W and Z bosons and, in the process, predicts the existence of a new particle: the Higgs boson \cite{pich2012standardmodelelectroweakinteractions}.

For decades, the Higgs boson remained the missing piece of the SM. In 2012, experiments at CERN’s Large Hadron Collider (LHC), conducted by the ATLAS and CMS teams, discovered a new particle with a mass of about 125 GeV \cite{Chatrchyan_2012}. This particle’s properties have been studied extensively and align with the Higgs boson predicted by the SM, confirming the mechanism that gives mass to other particles. Today, the Higgs boson’s mass is measured to be approximately 125 GeV, with only a small uncertainty \cite{Chatrchyan_2012}.

Despite its success in explaining many experimental results, the SM is not a complete theory. It leaves several big questions unanswered. The SM does not explain why there are exactly three generations of particles, each with similar properties but different masses. It also originally assumed neutrinos had no mass, but by now it is know they have tiny masses, meaning the model needs to be adjusted.
The SM does not solve the mystery of why the Higgs boson’s mass is so small compared to what quantum effects suggest either - a problem called the hierarchy problem. These gaps suggest the SM is just part of a bigger picture that has not been fully uncovered yet. To do so, particle accelerators are used to challenge the assumptions made in the SM and even go farther - or in this case smaller and into even higher energy scales - and try to look for physics beyond the SM.

\section{The Large Hadron Collider}

To explore phenomena at distance scales far below $10^{-18}$ m (i.e. at extremely high energy scales), physicists rely on high-energy particle collisions. The LHC at CERN is the world’s largest and most powerful particle accelerator, designed to probe such distance scales by colliding protons at unprecedented energies. It is a 27 km circumference circular accelerator that accelerates two counter-rotating beams of protons to nearly the speed of light and brings them into head-on collision at four interaction points. The LHC, which design centre-of-mass energy is 14 TeV (7 TeV per beam) \cite{Evans:2008zzb}, was built to achieve a peak instantaneous luminosity of $1\times10^{34}$ cm$^{-2}$s$^{-1}$ \cite{Evans:2008zzb}. In its successful Run-1 (2010–2013) and Run-2 (2015–2018) operations, the LHC reached collision energies up to 13 TeV and even exceeded the design luminosity – achieving about $2\times10^{34}$ cm$^{-2}$s$^{-1}$ in 2018 \cite{Hayrapetyan_2024}. The machine is now in Run-3 (2022–present) with a slight energy increase (13.6 TeV). It is being upgraded for the High-Luminosity LHC (HL-LHC) era later this decade, which aims to further boost the luminosity by about an order of magnitude.

\begin{figure}[h]
    \centering
    \includegraphics[width=0.9\linewidth]{media/CCC-v2018-print-v2.pdf}
    \caption{ Sketch of the CERN accelerator complex, adapted from \cite{Mobs:2636343}.}
    \label{fig:lhc}
\end{figure}

Figure \ref{fig:lhc} shows the structure of the CERN complex with the LHC at its heart. On the discovery front, the LHC’s first triumphant success was the Higgs boson. Ongoing searches continue for signs of physics beyond the Standard Model – for example, heavy supersymmetric particles, extra dimensions, or new force carriers – across many possible decay channels. So far, no clear evidence of new particles has appeared; extensive searches in the most promising channels have found no statistically significant deviations from SM expectations \cite{sonneveld2025susyhighlightscurrentresults}.

\section{The CMS Experiment}

Collision events in the LHC produce quarks and gluons that cannot exist as free particles (colour confinement \cite{pich2012standardmodelelectroweakinteractions}). Instead, the outgoing parton radiates additional quarks and gluons, forming a parton shower whose virtuality cascades down to $\mathcal{O}(\mathrm{GeV})$, where hadronisation binds them into colour-neutral hadrons. Because soft and collinear emissions are enhanced in QCD, these hadrons emerge in a narrow cone around the original parton direction, forming a visible spray—a jet. The CMS detector is one of the two general-purpose detectors at the LHC designed to record these jets with high efficiency and precision, enabling a broad physics program from Higgs boson studies to searches for new phenomena. 

The CMS detector is built in a layered, cylindrical geometry around the collision point (see figure \ref{fig:cms_overview}) and, despite the name "Compact", enormous in absolute terms: it is about 15~m in diameter, 21~m in length, and weighs about 14,000 tonnes. At the heart of CMS is a superconducting solenoid coil that generates a magnetic field of 3.8 T within a 6~m inner diameter. This bends the trajectories of charged particles, allowing their momenta to be measured; the steel structure that contains the magnetic flux (the return yoke) also serves as the absorber for muon detection and accounts for the bulk of CMS’s mass. Inside the magnet coil, CMS is packed with high-precision tracking and calorimetry systems, and outside the coil are large muon detector chambers, as shown in figure \ref{fig:cms_overview} \cite{CMS}.

\begin{figure}[h]
\centering
    \includegraphics[width=15cm]{media/cms_cutview.png}
    \caption{CMS Detector adapted from \cite{Sirunyan_2017}.}
    \label{fig:cms_overview}
\end{figure}

Moving outwards from the beam line, the first subsystem is the silicon tracker, a high-granularity detector made of about 75 million individual silicon strips and pixels arranged in concentric layers \cite{CMS}. When a charged particle from a collision passes through the tracker, it leaves hits in these silicon sensors. By reconstructing sequences of hits, CMS can trace out the tracks of charged particles with fine spatial resolution. The strong magnetic field bends these tracks; the particle’s momentum can be determined from the curvature. The tracker is designed to be extremely precise, allowing the reconstruction of secondary vertices from $b$-hadron decays a few centimetres from the collision point.

Surrounding the tracker is the electromagnetic calorimeter (ECAL), which is made of dense lead tungstate ($\text{PbWO}_4$) crystals. The ECAL’s task is to fully absorb and measure the energy of electrons and photons \cite{CMS}. When an electron or high-energy photon enters the ECAL, it initiates an electromagnetic shower in the crystal. The light output from the crystals is proportional to the particle’s energy. The CMS ECAL provides excellent energy resolution (on the order of 1\% for high-energy electrons/photons) and was pivotal in the Higgs boson discovery via the $H\to\gamma\gamma$ decay mode \cite{Chatrchyan_2012}.

Outside the ECAL lies the hadron calorimeter (HCAL). It is sampling calorimeter using alternating layers of absorber (brass or steel) and plastic scintillator. The HCAL is designed to stop and measure the energy of hadrons (particles made of quarks, such as pions, kaons, protons, etc.) \cite{CMS}. Hadrons penetrate the ECAL but are largely absorbed in the thicker material of the HCAL, producing cascades of secondary particles. By collecting the scintillation light from these showers, the HCAL provides a measurement of the hadronic energy. Although the HCAL’s resolution is coarser than that of the ECAL, combining its information with the ECAL and tracker allows reconstruction of the energy and direction of jets as well as the estimation of missing transverse energy.

The outermost layers of CMS are the dedicated muon detectors, which give the experiment its name. Muons are charged leptons similar to electrons but about 200 times heavier, and they are penetrating: unlike most particles, muons can traverse substantial amounts of matter. In CMS, after passing through the calorimeters, muons still have enough energy to reach the muon chambers interleaved in the steel return yoke \cite{CMS}. CMS employs several types of muon detectors (drift tubes, cathode strip chambers, and resistive plate chambers) to track muons independently of the inner tracker. By matching muon tracks in the muon system with those in the inner tracker, CMS achieves a very accurate muon momentum measurement.

\subsection{Trigger System}
As the LHC collision rate is enormous, CMS cannot record data from every collision. Instead, the experiment uses a trigger system to filter events in real time. CMS employs a two-level trigger system: a Level-1 (L1) trigger implemented in custom electronics (fast hardware logic) and a High-Level Trigger (HLT) implemented in software running on a computing farm \cite{Khachatryan_2017}. The L1 trigger, which is using information from the calorimeters and muon system, reduces the 40 MHz collision rate to around 100 kHz by selecting events with interesting signatures (such as high-energy objects). Then the HLT takes those L1-selected events and runs a streamlined version of the full event reconstruction to apply more refined selection criteria, outputting a final rate of around 1 kHz to permanent storage \cite{Khachatryan_2017}. This multi-tiered trigger is crucial for ensuring that the most interesting collisions – those potentially containing rare new physics or useful signals – are recorded for offline analysis, while discarding the rest.

\subsection{Coordinates}
CMS uses a right-handed coordinate system with the origin at the centre of the detector. The x-axis points radially inward toward the centre of the LHC ring, the y-axis points vertically upward, and the z-axis is aligned with the beam direction (pointing towards the Jura mountains) \cite{Chatrychan_2008}. Instead of simple Cartesian coordinates $(x,y,z)$, it is often convenient to use cylindrical and spherical coordinates $(r,\phi,\theta)$ (or equivalently $(r,\phi,\eta)$) to describe angles and distances. Here, $r$ denotes the distance from the beam line in the transverse plane ($x$–$y$ plane), $\phi$ is the azimuthal angle in that plane (with $\phi=0$ defined along the positive x-axis), and $\theta$ is the polar angle measured from the positive $z$-axis.

In practice, the polar angle $\theta$ is expressed via the pseudorapidity $\eta$, defined as:

\begin{equation}
    \eta = - \ln \left( \tan \frac{\theta}{2}\right)
\end{equation}

For highly relativistic particles (with $E \gg m$), the pseudorapidity $\eta$ is approximately equal to the particle’s rapidity $y$, which is given by

\begin{equation}
    y = \frac{1}{2} \ln\frac{E+p_z}{E-p_z},
\end{equation}

where $E$ is the particle’s energy and $p_z$ is the $z$-component of the momentum; the transverse momentum is $p_T=\sqrt{p_x^2+p_y^2}$. This coordinate choice conveniently captures the detector’s cylindrical symmetry and the boost-invariant nature of motion along the beam axis \cite{Chatrychan_2008}.

% Chapter 2
\chapter{Machine Learning in High Energy Physics}
Lorem ipsum set dolor.

\section{Basic Concepts}
Lorem ipsum set dolor.

\section{Adversarial Machine Learning}
Lorem ipsum set dolor.

\subsection{Adversarial Attacks}
Lorem ipsum set dolor.

\subsection{Challenges in Structured and Discrete Input Spaces}
Lorem ipsum set dolor.

\section{The Deep Jet Tagger}
Lorem ipsum set dolor.

% Chapter 3
\chapter{Methodology}

\section{Initial Epxloration of PDG}
Lorem ipsum set dolor.

\subsection{PGD on Continious Inputs}
Lorem ipsum set dolor.

\subsection{PGD with naive Integer Handling}
Lorem ipsum set dolor.

\section{Probabilistic Integer Pertubation (IntProb)}
Lorem ipsum set dolor.

\section{Combination of PGD and IntProb}
Lorem ipsum set dolor.

\section{Implementation Details}
Lorem ipsum set dolor.

\subsection{Feature Relevance and Input Clamping}
Lorem ipsum set dolor.

\subsection{Integration into DeepJet Framework}
Lorem ipsum set dolor.

\section{Evaluation Metrics and Threat Model}
Lorem ipsum set dolor.


% Chapter 4
\chapter{Adversarial Studies on DeepJet}
Lorem ipsum set dolor.

\section{Adversarial Attacks on DeepJet}
Lorem ipsum set dolor.

\section{Adversarial Training on DeepJet}
Lorem ipsum set dolor.

\section{Transferability and Cross-Robustness}
Lorem ipsum set dolor.

% Chapter 5
\chapter{Conclusion and Outlook}

This thesis has introduced PIP, specifically designed for discrete features in particle physics applications. Applied to the DeepJet flavour tagging algorithm used in CMS, this work demonstrates how to develop a robustness against vulnerabilities in state-of-the-art machine learning models that were previously disregarded.

The key innovation of PIP comes from its probabilistic approach to discrete perturbations without continuous relaxation. Unlike continuous attack methods that generate non-physical fractional values when being applied to integer features, PIP uses gradient information to probabilistically determine whether to increment or decrement discrete features by integer units. This ensures all perturbations remain physically meaningful while effectively probing model robustness. A tunable sharpness parameter controls the attack's aggressiveness, allowing researchers to balance effectiveness with stealth considerations.

The investigation of PIP-PGD attacks reveals synergistic effects exceeding the impact of either method alone. Attacking both, continuous and discrete, features simultaneously exploits complementary vulnerabilities, leading to more severe performance degradation than expected from individual attack impacts.

Adversarial training experiments show that models trained against only one attack type exhibit limited transferability to other attack types, indicating attack-specific overfitting. However, adversarial training with PIP-PGD attacks provides balanced robustness across all tested scenarios while maintaining nominal performance within 1-2\% of baseline levels. This combined training approach offers the most comprehensive protection against both continuous and discrete adversarial perturbations.

Beyond the specific application to DeepJet, this work contributes to the broader field of adversarial machine learning by suggesting a general and computational cheap framework for attacking discrete features on top of continuous values that can be adapted to other scientific domains. The successful combination of continuous and discrete attacks demonstrates the importance of considering multiple perturbation modalities in robustness evaluation, particularly relevant for scientific applications involving mixed data types.

\newpage

While PIP offers a robust method that can be applied in conjunction with existing continuous attack methods, several improvements could enhance its effectiveness. The current implementation does not account for relative feature scaling in the same way that PGD uses feature-specific scaling factors. Incorporating relative scaling based on feature importance or sensitivity has the potential to enhance the efficacy while maintaining physical realism.

The synergy between PGD and PIP could may be attributable to both methods acting on simple gradient information that are generally aligned. However, it remains uncertain whether PIP works equally well with other more sophisticated attack methods. Future research could explore integration with advanced continuous attacks to reveal new synergistic opportunities.

Beyond methodological improvements, PIP could be extended to other scientific machine learning applications where discrete features play important roles. Research areas such as astronomy, materials science, and bioinformatics often involve mixed data types that could benefit from similar robustness analysis approaches. Additionally, incorporating domain-specific knowledge about particle physics has the potential to create more realistic and targeted adversarial perturbations, improving both attack effectiveness and physical interpretability.

\printbibliography

\appendix

% Exclude appendix figures from list of illustrations
\let\oldaddcontentsline\addcontentsline
\renewcommand{\addcontentsline}[3]{%
  \ifnum\pdfstrcmp{#2}{figure}=0
    \ifnum\value{chapter}>0
      % Don't add figures to list of illustrations in appendix
    \else
      \oldaddcontentsline{#1}{#2}{#3}%
    \fi
  \else
    \oldaddcontentsline{#1}{#2}{#3}%
  \fi
}

\chapter{Appendix}

\begin{table}[ht]
\centering
\caption{Evaluation metrics used to quantify model performance and perturbation impact.}
\begin{tabularx}{\textwidth}{@{} l X X @{}}
\toprule
\textbf{Metric} & \textbf{Definition / Calculation} & \textbf{Physical Interpretation} \\
\midrule
\textbf{ROC / AUC} &
The Receiver Operating Characteristic (ROC) curve plots the True Positive Rate (TPR, or sensitivity) against the False Positive Rate (FPR) across varying classification thresholds. The Area Under the Curve (AUC) is the integral of the ROC curve, ranging from 0.5 (random classifier) to 1.0 (perfect classifier). &
The ROC curve illustrates the trade-off between correctly identifying heavy-flavour jets (TPR) and mistakenly tagging background jets (FPR). A higher AUC indicates better model performance in distinguishing signal from background, critical for robust jet tagging. \\
\addlinespace
\textbf{\(\Delta\)AUC (AUC Drop)} &
Difference in the model’s Area Under the ROC Curve (AUC) between nominal (unperturbed) and adversarial datasets. Often reported as a percentage of the nominal AUC. &
Measures the loss of discriminative power for jet tagging. A larger drop means the adversarial attack significantly degrades the classifier’s ability to distinguish heavy-flavour jets from background (attack success). \\
\addlinespace
\textbf{Jensen–Shannon Distance (JSD)} &
The square root of the Jensen–Shannon divergence between the distribution of a given feature for nominal vs. attacked samples (see Equation~(\ref{eq:jsd})). Computed on each feature’s 1D histogram, excluding default/padding values. &
Quantifies the \emph{statistical difference} between original and perturbed feature distributions. Low JSD (near 0) indicates minimal change (stealthy), while high JSD reveals visible distribution shifts. \\
\hline
\end{tabularx}
\label{tab:evaluation_metrics}
\end{table}



\begin{figure}[h]
\centering
    \includegraphics[width=13cm]{media/output/intpgd_loss_validation.pdf}
    \caption{Training and validation loss for FGSM applied to the entire input spectrum training.}
    \label{fig:intpgd_train}
\end{figure}

\section{Input Features}

% Table for global input features
\begin{table}[h]
\centering
\caption{Descriptions of the global input features with applied masks $\mathcal{M_{\text{float}}}$, $\mathcal{M_{\text{int}}}$, $\mathcal{M_{\text{int}}^*\subset M_{\text{int}}}$.}
\begin{tabularx}{\textwidth}{|c|c|X|c|}
\hline
\textbf{variable name} & \textbf{type} & \textbf{description} & \textbf{mask} \\
\hline
\texttt{jet\_pt} & float & transverse momentum of the jet in GeV & \cellcolor{green!50}$\mathcal{M_{\text{float}}}$ \\
\hline
\texttt{jet\_eta} & float & pseudorapidity of the jet & \cellcolor{green!50}$\mathcal{M_{\text{float}}}$ \\
\hline
\texttt{n\_Cpfcand} & int & number of charged PF candidates in the jet & \cellcolor{orange!50}$\mathcal{M_{\text{int}}}$ \\
\hline
\texttt{n\_Npfcand} & int & number of neutral PF candidates in the jet & \cellcolor{orange!50}$\mathcal{M_{\text{int}}}$ \\
\hline
\texttt{nsv} & int & number of secondary vertices in the jet & \cellcolor{orange!50}$\mathcal{M_{\text{int}}}$ \\
\hline
\texttt{npv} & int & number of primary vertices in the event & \cellcolor{orange!50}$\mathcal{M_{\text{int}}}$ \\
\hline
\texttt{TagVarCSV\_trackSumJetEtRatio} & float & transverse energy ratio of tracks and jet & \cellcolor{green!50}$\mathcal{M_{\text{float}}}$ \\
\hline
\texttt{TagVarCSV\_trackSumJetDeltaR} & float & spatial distance between the sum of tracks and jet axis & \cellcolor{green!50}$\mathcal{M_{\text{float}}}$ \\
\hline
\texttt{TagVarCSV\_vertexCategory} & int & secondary vertex category index & \cellcolor{orange!50}$\mathcal{M_{\text{int}}}$ \\
\hline
\texttt{TagVarCSV\_trackSip2dValAboveCharm} & float & 2D signed impact parameter of first track lifting mass above charm in cm & \cellcolor{green!50}$\mathcal{M_{\text{float}}}$ \\
\hline
\texttt{TagVarCSV\_trackSip2dSigAboveCharm} & float & 2D signed impact parameter significance of first track lifting mass above charm & \cellcolor{green!50}$\mathcal{M_{\text{float}}}$ \\
\hline
\texttt{TagVarCSV\_trackSip3dValAboveCharm} & float & 3D signed impact parameter of first track lifting mass above charm in cm & \cellcolor{green!50}$\mathcal{M_{\text{float}}}$ \\
\hline
\texttt{TagVarCSV\_trackSip3dSigAboveCharm} & float & 3D signed impact parameter significance of first track lifting mass above charm & \cellcolor{green!50}$\mathcal{M_{\text{float}}}$ \\
\hline
\texttt{TagVarCSV\_jetNSelectedTracks} & int & number of tracks associated to jet & \cellcolor{orange!50}$\mathcal{M_{\text{int}}}$ \\
\hline
\texttt{TagVarCSV\_jetNTracksEtaRel} & int & number of tracks associated to jet for which \( \eta_{\text{rel}} \) is computed & \cellcolor{orange!50}$\mathcal{M_{\text{int}}}$ \\
\hline
\end{tabularx}
\label{tab:global_input_features}
\end{table}


% Table for CPF input features
\begin{table}[h]
\centering
\caption{Descriptions of the CPF input features with applied masks $\mathcal{M_{\text{float}}}$, $\mathcal{M_{\text{int}}}$, $\mathcal{M_{\text{int}}^*\subset M_{\text{int}}}$.}
\begin{tabularx}{\textwidth}{|c|c|X|c|}
\hline
\textbf{variable name} & \textbf{type} & \textbf{description} & \textbf{mask} \\
\hline
\texttt{Cpfcan\_BtagPf\_trackEtaRel} & float & pseudorapidity of charged track relative to jet axis & \cellcolor{green!50}$\mathcal{M_{\text{float}}}$ \\
\hline
\texttt{Cpfcan\_BtagPf\_trackPtRel} & float & momentum of charged track transverse to jet axis in GeV & \cellcolor{green!50}$\mathcal{M_{\text{float}}}$ \\
\hline
\texttt{Cpfcan\_BtagPf\_trackPPar} & float & momentum of charged track along jet axis in GeV & \cellcolor{green!50}$\mathcal{M_{\text{float}}}$ \\
\hline
\texttt{Cpfcan\_BtagPf\_trackDeltaR} & float & spatial distance between track and jet axis & \cellcolor{green!50}$\mathcal{M_{\text{float}}}$ \\
\hline
\texttt{Cpfcan\_BtagPf\_trackPParRatio} & float & momentum fraction of charged track along jet axis & \cellcolor{green!50}$\mathcal{M_{\text{float}}}$ \\
\hline
\texttt{Cpfcan\_BtagPf\_trackSip2dVal} & float & 2D signed impact parameter of charged track in cm & \cellcolor{green!50}$\mathcal{M_{\text{float}}}$ \\
\hline
\texttt{Cpfcan\_BtagPf\_trackSip2dSig} & float & 2D signed impact parameter significance of charged track & \cellcolor{green!50}$\mathcal{M_{\text{float}}}$ \\
\hline
\texttt{Cpfcan\_BtagPf\_trackSip3dVal} & float & 3D signed impact parameter of charged track in cm & \cellcolor{green!50}$\mathcal{M_{\text{float}}}$ \\
\hline
\texttt{Cpfcan\_BtagPf\_trackSip3dSig} & float & 3D signed impact parameter significance of charged track & \cellcolor{green!50}$\mathcal{M_{\text{float}}}$ \\
\hline
\texttt{Cpfcan\_BtagPf\_trackJetDistVal} & float & minimum distance between track and jet axis & \cellcolor{green!50}$\mathcal{M_{\text{float}}}$ \\
\hline
\texttt{Cpfcan\_ptrel} & float & transverse momentum fraction of the track & \cellcolor{green!50}$\mathcal{M_{\text{float}}}$ \\
\hline
\texttt{Cpfcan\_drminsv} & float & spatial distance between the track and the closest secondary vertex & \cellcolor{green!50}$\mathcal{M_{\text{float}}}$ \\
\hline
\texttt{Cpfcan\_VTX\_ass} & int & integer flag: usage of track in primary vertex fit & \cellcolor{orange!50}$\mathcal{M_{\text{int}}}^*$ \\
\hline
\texttt{Cpfcan\_puppiw} & int & PUPPI weight of the charged PF candidate & \cellcolor{red!50}$\mathcal{M_{\text{int}}}$ \\
\hline
\texttt{Cpfcan\_chi2} & int & \( \chi^2 \) of charged track fit & \cellcolor{red!50}$\mathcal{M_{\text{int}}}$ \\
\hline
\texttt{Cpfcan\_quality} & int & integer flag: quality of fitted track & \cellcolor{orange!50}$\mathcal{M_{\text{int}}}^*$ \\
\hline
\end{tabularx}
\label{tab:cpf_input_features}
\end{table}

\newpage
\FloatBarrier
% Table for SV input features
\begin{table}[H]
% Table for NPF input features
\centering
\caption{Descriptions of the NPF input features with applied masks $\mathcal{M_{\text{float}}}$, $\mathcal{M_{\text{int}}}$, $\mathcal{M_{\text{int}}^*\subset M_{\text{int}}}$.}
\begin{tabularx}{\textwidth}{|c|c|X|c|}
\hline
\textbf{variable name} & \textbf{type} & \textbf{description} & \textbf{mask} \\
\hline
\texttt{Npfcan\_ptrel} & float & transverse momentum fraction of the track & \cellcolor{green!50}$\mathcal{M_{\text{float}}}$ \\
\hline
\texttt{Npfcan\_deltaR} & float & spatial distance between track and jet axis & \cellcolor{green!50}$\mathcal{M_{\text{float}}}$ \\
\hline
\texttt{Npfcan\_isGamma} & int & integer flag: particle identity (1 = photon, 0 = no photon) & \cellcolor{orange!50}$\mathcal{M_{\text{int}}}^*$ \\
\hline
\texttt{Npfcan\_HadFrac} & int & fraction of the energy deposited in the hadronic calorimeter & \cellcolor{orange!50}$\mathcal{M_{\text{int}}}^*$ \\
\hline
\texttt{Npfcan\_drminsv} & float & spatial distance between the track and the closest secondary vertex & \cellcolor{green!50}$\mathcal{M_{\text{float}}}$ \\
\hline
\texttt{Npfcan\_puppiw} & int & PUPPI weight of the neutral PF candidate & \cellcolor{red!50}$\mathcal{M_{\text{int}}}$ \\
\hline
\end{tabularx}
\label{tab:npf_input_features}
\end{table}

\begin{table}[h]
\centering
\caption{Descriptions of the SV input features with applied masks $\mathcal{M_{\text{float}}}$, $\mathcal{M_{\text{int}}}$, $\mathcal{M_{\text{int}}^*\subset M_{\text{int}}}$.}
\begin{tabularx}{\textwidth}{|c|c|X|c|}
\hline
\textbf{variable name} & \textbf{data type} & \textbf{description} & \textbf{mask} \\
\hline
\texttt{sv\_pt} & float & transverse momentum of the SV in GeV & \cellcolor{green!50}$\mathcal{M_{\text{float}}}$ \\
\hline
\texttt{sv\_deltaR} & float & spatial distance between SV and jet axis & \cellcolor{green!50}$\mathcal{M_{\text{float}}}$ \\
\hline
\texttt{sv\_mass} & float & mass of the SV in GeV & \cellcolor{green!50}$\mathcal{M_{\text{float}}}$ \\
\hline
\texttt{sv\_ntracks} & int & number of tracks associated to the SV & \cellcolor{red!50}$\mathcal{M_{\text{int}}}$ \\
\hline
\texttt{sv\_chi2} & int & \( \chi^2 \) of the SV fit & \cellcolor{red!50}$\mathcal{M_{\text{int}}}$ \\
\hline
\texttt{sv\_normchi2} & int & \( \chi^2 \) divided by the degrees of freedom & \cellcolor{red!50}$\mathcal{M_{\text{int}}}$ \\
\hline
\texttt{sv\_dxy} & float & 2D flight distance of SV in cm & \cellcolor{green!50}$\mathcal{M_{\text{float}}}$ \\
\hline
\texttt{sv\_dxysig} & float & 2D flight distance significance of SV & \cellcolor{green!50}$\mathcal{M_{\text{float}}}$ \\
\hline
\texttt{sv\_d3d} & float & 3D flight distance of SV in cm & \cellcolor{green!50}$\mathcal{M_{\text{float}}}$ \\
\hline
\texttt{sv\_d3dsig} & float & 3D flight distance significance of SV & \cellcolor{green!50}$\mathcal{M_{\text{float}}}$ \\
\hline
\texttt{sv\_costhetasvpv} & float & cosine of the angle between SV flight direction and SV momentum & \cellcolor{green!50}$\mathcal{M_{\text{float}}}$ \\
\hline
\texttt{sv\_enratio} & float & energy fraction of the SV & \cellcolor{green!50}$\mathcal{M_{\text{float}}}$ \\
\hline
\end{tabularx}
\label{tab:sv_input_features}
\end{table}


\section{PIP Input Similarities}
\label{appendix:intprob}

\subsection*{Global Features}

\begin{figure}[htbp]
  \centering
  \begin{subfigure}[t]{0.32\textwidth}
    \includegraphics[width=\linewidth]{media/output/features/compare/intprob_1/cmp_global_features_n_Cpfcand.pdf}
    \caption{Input similarity for PIP(1).}
  \end{subfigure}\hfill
  \begin{subfigure}[t]{0.32\textwidth}
    \includegraphics[width=\linewidth]{media/output/features/compare/intprob_2/cmp_global_features_n_Cpfcand.pdf}
    \caption{Input similarity for PIP(2).}
  \end{subfigure}\hfill
  \begin{subfigure}[t]{0.32\textwidth}
    \includegraphics[width=\linewidth]{media/output/features/compare/intprob_3/cmp_global_features_n_Cpfcand.pdf}
    \caption{Input similarity for PIP(3).}
  \end{subfigure}

  \caption{Histogram for \texttt{n\_Cpfcand} for multiple iterations of PIP tested against nominal inputs.}
  \label{fig:intprob_input_n_Cpfcand}
\end{figure}
\begin{figure}[htbp]
  \centering
  \begin{subfigure}[t]{0.32\textwidth}
    \includegraphics[width=\linewidth]{media/output/features/compare/intprob_1/cmp_global_features_n_Npfcand.pdf}
    \caption{Input similarity for PIP(1).}
  \end{subfigure}\hfill
  \begin{subfigure}[t]{0.32\textwidth}
    \includegraphics[width=\linewidth]{media/output/features/compare/intprob_2/cmp_global_features_n_Npfcand.pdf}
    \caption{Input similarity for PIP(2).}
  \end{subfigure}\hfill
  \begin{subfigure}[t]{0.32\textwidth}
    \includegraphics[width=\linewidth]{media/output/features/compare/intprob_3/cmp_global_features_n_Npfcand.pdf}
    \caption{Input similarity for PIP(3).}
  \end{subfigure}

  \caption{Histogram for \texttt{n\_Npfcand} for multiple iterations of PIP tested against nominal inputs.}
  \label{fig:intprob_input_n_Npfcand}
\end{figure}
\begin{figure}[htbp]
  \centering
  \begin{subfigure}[t]{0.32\textwidth}
    \includegraphics[width=\linewidth]{media/output/features/compare/intprob_1/cmp_global_features_npv.pdf}
    \caption{Input similarity for PIP(1).}
  \end{subfigure}\hfill
  \begin{subfigure}[t]{0.32\textwidth}
    \includegraphics[width=\linewidth]{media/output/features/compare/intprob_2/cmp_global_features_npv.pdf}
    \caption{Input similarity for PIP(2).}
  \end{subfigure}\hfill
  \begin{subfigure}[t]{0.32\textwidth}
    \includegraphics[width=\linewidth]{media/output/features/compare/intprob_3/cmp_global_features_npv.pdf}
    \caption{Input similarity for PIP(3).}
  \end{subfigure}

  \caption{Histogram for \texttt{npv} for multiple iterations of PIP tested against nominal inputs.}
  \label{fig:intprob_input_npv}
\end{figure}
\begin{figure}[htbp]
  \centering
  \begin{subfigure}[t]{0.32\textwidth}
    \includegraphics[width=\linewidth]{media/output/features/compare/intprob_1/cmp_global_features_nsv.pdf}
    \caption{Input similarity for PIP(1).}
  \end{subfigure}\hfill
  \begin{subfigure}[t]{0.32\textwidth}
    \includegraphics[width=\linewidth]{media/output/features/compare/intprob_2/cmp_global_features_nsv.pdf}
    \caption{Input similarity for PIP(2).}
  \end{subfigure}\hfill
  \begin{subfigure}[t]{0.32\textwidth}
    \includegraphics[width=\linewidth]{media/output/features/compare/intprob_3/cmp_global_features_nsv.pdf}
    \caption{Input similarity for PIP(3).}
  \end{subfigure}

  \caption{Histogram for \texttt{nsv} for multiple iterations of PIP tested against nominal inputs.}
  \label{fig:intprob_input_nsv}
\end{figure}
\begin{figure}[htbp]
  \centering
  \begin{subfigure}[t]{0.32\textwidth}
    \includegraphics[width=\linewidth]{media/output/features/compare/intprob_1/cmp_global_features_TagVarCSV_jetNSelectedTracks.pdf}
    \caption{Input similarity for PIP(1).}
  \end{subfigure}\hfill
  \begin{subfigure}[t]{0.32\textwidth}
    \includegraphics[width=\linewidth]{media/output/features/compare/intprob_2/cmp_global_features_TagVarCSV_jetNSelectedTracks.pdf}
    \caption{Input similarity for PIP(2).}
  \end{subfigure}\hfill
  \begin{subfigure}[t]{0.32\textwidth}
    \includegraphics[width=\linewidth]{media/output/features/compare/intprob_3/cmp_global_features_TagVarCSV_jetNSelectedTracks.pdf}
    \caption{Input similarity for PIP(3).}
  \end{subfigure}

  \caption{Histogram for \texttt{TagVarCSV\_jetNSelectedTracks} for multiple iterations of PIP tested against nominal inputs.}
  \label{fig:intprob_input_TagVarCSV_jetNSelectedTracks}
\end{figure}
\begin{figure}[htbp]
  \centering
  \begin{subfigure}[t]{0.32\textwidth}
    \includegraphics[width=\linewidth]{media/output/features/compare/intprob_1/cmp_global_features_TagVarCSV_jetNTracksEtaRel.pdf}
    \caption{Input similarity for PIP(1).}
  \end{subfigure}\hfill
  \begin{subfigure}[t]{0.32\textwidth}
    \includegraphics[width=\linewidth]{media/output/features/compare/intprob_2/cmp_global_features_TagVarCSV_jetNTracksEtaRel.pdf}
    \caption{Input similarity for PIP(2).}
  \end{subfigure}\hfill
  \begin{subfigure}[t]{0.32\textwidth}
    \includegraphics[width=\linewidth]{media/output/features/compare/intprob_3/cmp_global_features_TagVarCSV_jetNTracksEtaRel.pdf}
    \caption{Input similarity for PIP(3).}
  \end{subfigure}

  \caption{Histogram for \texttt{TagVarCSV\_jetNTracksEtaRel} for multiple iterations of PIP tested against nominal inputs.}
  \label{fig:intprob_input_TagVarCSV_jetNTracksEtaRel}
\end{figure}
\begin{figure}[htbp]
  \centering
  \begin{subfigure}[t]{0.32\textwidth}
    \includegraphics[width=\linewidth]{media/output/features/compare/intprob_1/cmp_global_features_TagVarCSV_vertexCategory.pdf}
    \caption{Input similarity for PIP(1).}
  \end{subfigure}\hfill
  \begin{subfigure}[t]{0.32\textwidth}
    \includegraphics[width=\linewidth]{media/output/features/compare/intprob_2/cmp_global_features_TagVarCSV_vertexCategory.pdf}
    \caption{Input similarity for PIP(2).}
  \end{subfigure}\hfill
  \begin{subfigure}[t]{0.32\textwidth}
    \includegraphics[width=\linewidth]{media/output/features/compare/intprob_3/cmp_global_features_TagVarCSV_vertexCategory.pdf}
    \caption{Input similarity for PIP(3).}
  \end{subfigure}

  \caption{Histogram for \texttt{TagVarCSV\_vertexCategory} for multiple iterations of PIP tested against nominal inputs.}
  \label{fig:intprob_input_TagVarCSV_vertexCategory}
\end{figure}

\newpage
\subsection*{CPF Features}

\begin{figure}[htbp]
  \centering
  \begin{subfigure}[t]{0.32\textwidth}
    \includegraphics[width=\linewidth]{media/output/features/compare/intprob_1/cmp_cpf_arr_Cpfcan_quality.pdf}
    \caption{Input similarity for PIP(1).}
  \end{subfigure}\hfill
  \begin{subfigure}[t]{0.32\textwidth}
    \includegraphics[width=\linewidth]{media/output/features/compare/intprob_2/cmp_cpf_arr_Cpfcan_quality.pdf}
    \caption{Input similarity for PIP(2).}
  \end{subfigure}\hfill
  \begin{subfigure}[t]{0.32\textwidth}
    \includegraphics[width=\linewidth]{media/output/features/compare/intprob_3/cmp_cpf_arr_Cpfcan_quality.pdf}
    \caption{Input similarity for PIP(3).}
  \end{subfigure}

  \caption{Histogram for \texttt{Cpfcan\_quality} for multiple iterations of PIP tested against nominal inputs.}
  \label{fig:intprob_input_Cpfcan_quality}
\end{figure}
\begin{figure}[htbp]
  \centering
  \begin{subfigure}[t]{0.32\textwidth}
    \includegraphics[width=\linewidth]{media/output/features/compare/intprob_1/cmp_cpf_arr_Cpfcan_VTX_ass.pdf}
    \caption{Input similarity for PIP(1).}
  \end{subfigure}\hfill
  \begin{subfigure}[t]{0.32\textwidth}
    \includegraphics[width=\linewidth]{media/output/features/compare/intprob_2/cmp_cpf_arr_Cpfcan_VTX_ass.pdf}
    \caption{Input similarity for PIP(2).}
  \end{subfigure}\hfill
  \begin{subfigure}[t]{0.32\textwidth}
    \includegraphics[width=\linewidth]{media/output/features/compare/intprob_3/cmp_cpf_arr_Cpfcan_VTX_ass.pdf}
    \caption{Input similarity for PIP(3).}
  \end{subfigure}

  \caption{Histogram for \texttt{Cpfcan\_VTX\_ass} for multiple iterations of PIP tested against nominal inputs.}
  \label{fig:intprob_input_Cpfcan_VTX_ass}
\end{figure}

\newpage
\subsection*{NPF Features}

\begin{figure}[htbp]
  \centering
  \begin{subfigure}[t]{0.32\textwidth}
    \includegraphics[width=\linewidth]{media/output/features/compare/intprob_1/cmp_npf_arr_Npfcan_HadFrac.pdf}
    \caption{Input similarity for PIP(1).}
  \end{subfigure}\hfill
  \begin{subfigure}[t]{0.32\textwidth}
    \includegraphics[width=\linewidth]{media/output/features/compare/intprob_2/cmp_npf_arr_Npfcan_HadFrac.pdf}
    \caption{Input similarity for PIP(2).}
  \end{subfigure}\hfill
  \begin{subfigure}[t]{0.32\textwidth}
    \includegraphics[width=\linewidth]{media/output/features/compare/intprob_3/cmp_npf_arr_Npfcan_HadFrac.pdf}
    \caption{Input similarity for PIP(3).}
  \end{subfigure}

  \caption{Histogram for \texttt{Npfcan\_HadFrac} for multiple iterations of PIP tested against nominal inputs.}
  \label{fig:intprob_input_Npfcan_HadFrac}
\end{figure}
\begin{figure}[htbp]
  \centering
  \begin{subfigure}[t]{0.32\textwidth}
    \includegraphics[width=\linewidth]{media/output/features/compare/intprob_1/cmp_npf_arr_Npfcan_isGamma.pdf}
    \caption{Input similarity for PIP(1).}
  \end{subfigure}\hfill
  \begin{subfigure}[t]{0.32\textwidth}
    \includegraphics[width=\linewidth]{media/output/features/compare/intprob_2/cmp_npf_arr_Npfcan_isGamma.pdf}
    \caption{Input similarity for PIP(2).}
  \end{subfigure}\hfill
  \begin{subfigure}[t]{0.32\textwidth}
    \includegraphics[width=\linewidth]{media/output/features/compare/intprob_3/cmp_npf_arr_Npfcan_isGamma.pdf}
    \caption{Input similarity for PIP(3).}
  \end{subfigure}

  \caption{Histogram for \texttt{Npfcan\_isGamma} for multiple iterations of PIP tested against nominal inputs.}
  \label{fig:intprob_input_Npfcan_isGamma}
\end{figure}

\section{PIP-PGD Input Similarities}
\label{appendix:combined}

\subsection*{Global Features}


\begin{figure}[htbp]
  \centering
  \begin{subfigure}[t]{0.32\textwidth}
    \includegraphics[width=\linewidth]{media/output/features/compare/combined_it_1/cmp_global_features_jet_eta.pdf}
    \caption*{Input similarity for PIP-PGD(1).}
  \end{subfigure}\hfill
  \begin{subfigure}[t]{0.32\textwidth}
    \includegraphics[width=\linewidth]{media/output/features/compare/combined_it_2/cmp_global_features_jet_eta.pdf}
    \caption*{Input similarity for PIP-PGD(2).}
  \end{subfigure}\hfill
  \begin{subfigure}[t]{0.32\textwidth}
    \includegraphics[width=\linewidth]{media/output/features/compare/combined_it_3/cmp_global_features_jet_eta.pdf}
    \caption*{Input similarity for PIP-PGD(3).}
  \end{subfigure}

  \caption*{Histogram of \texttt{jet\_eta} for multiple iterations of PIP-PGD tested against nominal inputs.}
  \label{fig:combined_input_jet_eta}
\end{figure}

\begin{figure}[htbp]
  \centering
  \begin{subfigure}[t]{0.32\textwidth}
    \includegraphics[width=\linewidth]{media/output/features/compare/combined_it_1/cmp_global_features_jet_pt.pdf}
    \caption*{Input similarity for PIP-PGD(1).}
  \end{subfigure}\hfill
  \begin{subfigure}[t]{0.32\textwidth}
    \includegraphics[width=\linewidth]{media/output/features/compare/combined_it_2/cmp_global_features_jet_pt.pdf}
    \caption*{Input similarity for PIP-PGD(2).}
  \end{subfigure}\hfill
  \begin{subfigure}[t]{0.32\textwidth}
    \includegraphics[width=\linewidth]{media/output/features/compare/combined_it_3/cmp_global_features_jet_pt.pdf}
    \caption*{Input similarity for PIP-PGD(3).}
  \end{subfigure}

  \caption*{Histogram of \texttt{jet\_pt} for multiple iterations of PIP-PGD tested against nominal inputs.}
  \label{fig:combined_input_jet_pt}
\end{figure}

\begin{figure}[htbp]
  \centering
  \begin{subfigure}[t]{0.32\textwidth}
    \includegraphics[width=\linewidth]{media/output/features/compare/combined_it_1/cmp_global_features_n_Cpfcand.pdf}
    \caption*{Input similarity for PIP-PGD(1).}
  \end{subfigure}\hfill
  \begin{subfigure}[t]{0.32\textwidth}
    \includegraphics[width=\linewidth]{media/output/features/compare/combined_it_2/cmp_global_features_n_Cpfcand.pdf}
    \caption*{Input similarity for PIP-PGD(2).}
  \end{subfigure}\hfill
  \begin{subfigure}[t]{0.32\textwidth}
    \includegraphics[width=\linewidth]{media/output/features/compare/combined_it_3/cmp_global_features_n_Cpfcand.pdf}
    \caption*{Input similarity for PIP-PGD(3).}
  \end{subfigure}

  \caption*{Histogram of \texttt{n\_Cpfcand} for multiple iterations of PIP-PGD tested against nominal inputs.}
  \label{fig:combined_input_n_Cpfcand}
\end{figure}

\begin{figure}[htbp]
  \centering
  \begin{subfigure}[t]{0.32\textwidth}
    \includegraphics[width=\linewidth]{media/output/features/compare/combined_it_1/cmp_global_features_n_Npfcand.pdf}
    \caption*{Input similarity for PIP-PGD(1).}
  \end{subfigure}\hfill
  \begin{subfigure}[t]{0.32\textwidth}
    \includegraphics[width=\linewidth]{media/output/features/compare/combined_it_2/cmp_global_features_n_Npfcand.pdf}
    \caption*{Input similarity for PIP-PGD(2).}
  \end{subfigure}\hfill
  \begin{subfigure}[t]{0.32\textwidth}
    \includegraphics[width=\linewidth]{media/output/features/compare/combined_it_3/cmp_global_features_n_Npfcand.pdf}
    \caption*{Input similarity for PIP-PGD(3).}
  \end{subfigure}

  \caption*{Histogram of \texttt{n\_Npfcand} for multiple iterations of PIP-PGD tested against nominal inputs.}
  \label{fig:combined_input_n_Npfcand}
\end{figure}

\begin{figure}[htbp]
  \centering
  \begin{subfigure}[t]{0.32\textwidth}
    \includegraphics[width=\linewidth]{media/output/features/compare/combined_it_1/cmp_global_features_npv.pdf}
    \caption*{Input similarity for PIP-PGD(1).}
  \end{subfigure}\hfill
  \begin{subfigure}[t]{0.32\textwidth}
    \includegraphics[width=\linewidth]{media/output/features/compare/combined_it_2/cmp_global_features_npv.pdf}
    \caption*{Input similarity for PIP-PGD(2).}
  \end{subfigure}\hfill
  \begin{subfigure}[t]{0.32\textwidth}
    \includegraphics[width=\linewidth]{media/output/features/compare/combined_it_3/cmp_global_features_npv.pdf}
    \caption*{Input similarity for PIP-PGD(3).}
  \end{subfigure}

  \caption*{Histogram of \texttt{npv} for multiple iterations of PIP-PGD tested against nominal inputs.}
  \label{fig:combined_input_npv}
\end{figure}

\begin{figure}[htbp]
  \centering
  \begin{subfigure}[t]{0.32\textwidth}
    \includegraphics[width=\linewidth]{media/output/features/compare/combined_it_1/cmp_global_features_nsv.pdf}
    \caption*{Input similarity for PIP-PGD(1).}
  \end{subfigure}\hfill
  \begin{subfigure}[t]{0.32\textwidth}
    \includegraphics[width=\linewidth]{media/output/features/compare/combined_it_2/cmp_global_features_nsv.pdf}
    \caption*{Input similarity for PIP-PGD(2).}
  \end{subfigure}\hfill
  \begin{subfigure}[t]{0.32\textwidth}
    \includegraphics[width=\linewidth]{media/output/features/compare/combined_it_3/cmp_global_features_nsv.pdf}
    \caption*{Input similarity for PIP-PGD(3).}
  \end{subfigure}

  \caption*{Histogram of \texttt{nsv} for multiple iterations of PIP-PGD tested against nominal inputs.}
  \label{fig:combined_input_nsv}
\end{figure}

\begin{figure}[htbp]
  \centering
  \begin{subfigure}[t]{0.32\textwidth}
    \includegraphics[width=\linewidth]{media/output/features/compare/combined_it_1/cmp_global_features_TagVarCSV_jetNSelectedTracks.pdf}
    \caption*{Input similarity for PIP-PGD(1).}
  \end{subfigure}\hfill
  \begin{subfigure}[t]{0.32\textwidth}
    \includegraphics[width=\linewidth]{media/output/features/compare/combined_it_2/cmp_global_features_TagVarCSV_jetNSelectedTracks.pdf}
    \caption*{Input similarity for PIP-PGD(2).}
  \end{subfigure}\hfill
  \begin{subfigure}[t]{0.32\textwidth}
    \includegraphics[width=\linewidth]{media/output/features/compare/combined_it_3/cmp_global_features_TagVarCSV_jetNSelectedTracks.pdf}
    \caption*{Input similarity for PIP-PGD(3).}
  \end{subfigure}

  \caption*{Histogram of \texttt{TagVarCSV\_jetNSelectedTracks} for multiple iterations of PIP-PGD tested against nominal inputs.}
  \label{fig:combined_input_TagVarCSV_jetNSelectedTracks}
\end{figure}

\begin{figure}[htbp]
  \centering
  \begin{subfigure}[t]{0.32\textwidth}
    \includegraphics[width=\linewidth]{media/output/features/compare/combined_it_1/cmp_global_features_TagVarCSV_jetNTracksEtaRel.pdf}
    \caption*{Input similarity for PIP-PGD(1).}
  \end{subfigure}\hfill
  \begin{subfigure}[t]{0.32\textwidth}
    \includegraphics[width=\linewidth]{media/output/features/compare/combined_it_2/cmp_global_features_TagVarCSV_jetNTracksEtaRel.pdf}
    \caption*{Input similarity for PIP-PGD(2).}
  \end{subfigure}\hfill
  \begin{subfigure}[t]{0.32\textwidth}
    \includegraphics[width=\linewidth]{media/output/features/compare/combined_it_3/cmp_global_features_TagVarCSV_jetNTracksEtaRel.pdf}
    \caption*{Input similarity for PIP-PGD(3).}
  \end{subfigure}

  \caption*{Histogram of \texttt{TagVarCSV\_jetNTracksEtaRel} for multiple iterations of PIP-PGD tested against nominal inputs.}
  \label{fig:combined_input_TagVarCSV_jetNTracksEtaRel}
\end{figure}

\begin{figure}[htbp]
  \centering
  \begin{subfigure}[t]{0.32\textwidth}
    \includegraphics[width=\linewidth]{media/output/features/compare/combined_it_1/cmp_global_features_TagVarCSV_trackSip2dSigAboveCharm.pdf}
    \caption*{Input similarity for PIP-PGD(1).}
  \end{subfigure}\hfill
  \begin{subfigure}[t]{0.32\textwidth}
    \includegraphics[width=\linewidth]{media/output/features/compare/combined_it_2/cmp_global_features_TagVarCSV_trackSip2dSigAboveCharm.pdf}
    \caption*{Input similarity for PIP-PGD(2).}
  \end{subfigure}\hfill
  \begin{subfigure}[t]{0.32\textwidth}
    \includegraphics[width=\linewidth]{media/output/features/compare/combined_it_3/cmp_global_features_TagVarCSV_trackSip2dSigAboveCharm.pdf}
    \caption*{Input similarity for PIP-PGD(3).}
  \end{subfigure}

  \caption*{Histogram of \texttt{TagVarCSV\_trackSip2dSigAboveCharm} for multiple iterations of PIP-PGD tested against nominal inputs.}
  \label{fig:combined_input_TagVarCSV_trackSip2dSigAboveCharm}
\end{figure}

\begin{figure}[htbp]
  \centering
  \begin{subfigure}[t]{0.32\textwidth}
    \includegraphics[width=\linewidth]{media/output/features/compare/combined_it_1/cmp_global_features_TagVarCSV_trackSumJetDeltaR.pdf}
    \caption*{Input similarity for PIP-PGD(1).}
  \end{subfigure}\hfill
  \begin{subfigure}[t]{0.32\textwidth}
    \includegraphics[width=\linewidth]{media/output/features/compare/combined_it_2/cmp_global_features_TagVarCSV_trackSumJetDeltaR.pdf}
    \caption*{Input similarity for PIP-PGD(2).}
  \end{subfigure}\hfill
  \begin{subfigure}[t]{0.32\textwidth}
    \includegraphics[width=\linewidth]{media/output/features/compare/combined_it_3/cmp_global_features_TagVarCSV_trackSumJetDeltaR.pdf}
    \caption*{Input similarity for PIP-PGD(3).}
  \end{subfigure}

  \caption*{Histogram of \texttt{TagVarCSV\_trackSumJetDeltaR} for multiple iterations of PIP-PGD tested against nominal inputs.}
  \label{fig:combined_input_TagVarCSV_trackSumJetDeltaR}
\end{figure}

\begin{figure}[htbp]
  \centering
  \begin{subfigure}[t]{0.32\textwidth}
    \includegraphics[width=\linewidth]{media/output/features/compare/combined_it_1/cmp_global_features_TagVarCSV_trackSip2dValAboveCharm.pdf}
    \caption*{Input similarity for PIP-PGD(1).}
  \end{subfigure}\hfill
  \begin{subfigure}[t]{0.32\textwidth}
    \includegraphics[width=\linewidth]{media/output/features/compare/combined_it_2/cmp_global_features_TagVarCSV_trackSip2dValAboveCharm.pdf}
    \caption*{Input similarity for PIP-PGD(2).}
  \end{subfigure}\hfill
  \begin{subfigure}[t]{0.32\textwidth}
    \includegraphics[width=\linewidth]{media/output/features/compare/combined_it_3/cmp_global_features_TagVarCSV_trackSip2dValAboveCharm.pdf}
    \caption*{Input similarity for PIP-PGD(3).}
  \end{subfigure}

  \caption*{Histogram of \texttt{TagVarCSV\_trackSip2dValAboveCharm} for multiple iterations of PIP-PGD tested against nominal inputs.}
  \label{fig:combined_input_TagVarCSV_trackSip2dValAboveCharm}
\end{figure}

\begin{figure}[htbp]
  \centering
  \begin{subfigure}[t]{0.32\textwidth}
    \includegraphics[width=\linewidth]{media/output/features/compare/combined_it_1/cmp_global_features_TagVarCSV_trackSip3dSigAboveCharm.pdf}
    \caption*{Input similarity for PIP-PGD(1).}
  \end{subfigure}\hfill
  \begin{subfigure}[t]{0.32\textwidth}
    \includegraphics[width=\linewidth]{media/output/features/compare/combined_it_2/cmp_global_features_TagVarCSV_trackSip3dSigAboveCharm.pdf}
    \caption*{Input similarity for PIP-PGD(2).}
  \end{subfigure}\hfill
  \begin{subfigure}[t]{0.32\textwidth}
    \includegraphics[width=\linewidth]{media/output/features/compare/combined_it_3/cmp_global_features_TagVarCSV_trackSip3dSigAboveCharm.pdf}
    \caption*{Input similarity for PIP-PGD(3).}
  \end{subfigure}

  \caption*{Histogram of \texttt{TagVarCSV\_trackSip3dSigAboveCharm} for multiple iterations of PIP-PGD tested against nominal inputs.}
  \label{fig:combined_input_TagVarCSV_trackSip3dSigAboveCharm}
\end{figure}

\begin{figure}[htbp]
  \centering
  \begin{subfigure}[t]{0.32\textwidth}
    \includegraphics[width=\linewidth]{media/output/features/compare/combined_it_1/cmp_global_features_TagVarCSV_trackSip3dValAboveCharm.pdf}
    \caption*{Input similarity for PIP-PGD(1).}
  \end{subfigure}\hfill
  \begin{subfigure}[t]{0.32\textwidth}
    \includegraphics[width=\linewidth]{media/output/features/compare/combined_it_2/cmp_global_features_TagVarCSV_trackSip3dValAboveCharm.pdf}
    \caption*{Input similarity for PIP-PGD(2).}
  \end{subfigure}\hfill
  \begin{subfigure}[t]{0.32\textwidth}
    \includegraphics[width=\linewidth]{media/output/features/compare/combined_it_3/cmp_global_features_TagVarCSV_trackSip3dValAboveCharm.pdf}
    \caption*{Input similarity for PIP-PGD(3).}
  \end{subfigure}

  \caption*{Histogram of \texttt{TagVarCSV\_trackSip3dValAboveCharm} for multiple iterations of PIP-PGD tested against nominal inputs.}
  \label{fig:combined_input_TagVarCSV_trackSip3dValAboveCharm}
\end{figure}

\begin{figure}[htbp]
  \centering
  \begin{subfigure}[t]{0.32\textwidth}
    \includegraphics[width=\linewidth]{media/output/features/compare/combined_it_1/cmp_global_features_TagVarCSV_trackSumJetDeltaR.pdf}
    \caption*{Input similarity for PIP-PGD(1).}
  \end{subfigure}\hfill
  \begin{subfigure}[t]{0.32\textwidth}
    \includegraphics[width=\linewidth]{media/output/features/compare/combined_it_2/cmp_global_features_TagVarCSV_trackSumJetDeltaR.pdf}
    \caption*{Input similarity for PIP-PGD(2).}
  \end{subfigure}\hfill
  \begin{subfigure}[t]{0.32\textwidth}
    \includegraphics[width=\linewidth]{media/output/features/compare/combined_it_3/cmp_global_features_TagVarCSV_trackSumJetDeltaR.pdf}
    \caption*{Input similarity for PIP-PGD(3).}
  \end{subfigure}

  \caption*{Histogram of \texttt{TagVarCSV\_trackSumJetDeltaR} for multiple iterations of PIP-PGD tested against nominal inputs.}
  \label{fig:combined_input_TagVarCSV_trackSumJetDeltaR}
\end{figure}

\begin{figure}[htbp]
  \centering
  \begin{subfigure}[t]{0.32\textwidth}
    \includegraphics[width=\linewidth]{media/output/features/compare/combined_it_1/cmp_global_features_TagVarCSV_trackSumJetEtRatio.pdf}
    \caption*{Input similarity for PIP-PGD(1).}
  \end{subfigure}\hfill
  \begin{subfigure}[t]{0.32\textwidth}
    \includegraphics[width=\linewidth]{media/output/features/compare/combined_it_2/cmp_global_features_TagVarCSV_trackSumJetEtRatio.pdf}
    \caption*{Input similarity for PIP-PGD(2).}
  \end{subfigure}\hfill
  \begin{subfigure}[t]{0.32\textwidth}
    \includegraphics[width=\linewidth]{media/output/features/compare/combined_it_3/cmp_global_features_TagVarCSV_trackSumJetEtRatio.pdf}
    \caption*{Input similarity for PIP-PGD(3).}
  \end{subfigure}

  \caption*{Histogram of \texttt{TagVarCSV\_trackSumJetEtRatio} for multiple iterations of PIP-PGD tested against nominal inputs.}
  \label{fig:combined_input_TagVarCSV_trackSumJetEtRatio}
\end{figure}

\begin{figure}[htbp]
  \centering
  \begin{subfigure}[t]{0.32\textwidth}
    \includegraphics[width=\linewidth]{media/output/features/compare/combined_it_1/cmp_global_features_TagVarCSV_vertexCategory.pdf}
    \caption*{Input similarity for PIP-PGD(1).}
  \end{subfigure}\hfill
  \begin{subfigure}[t]{0.32\textwidth}
    \includegraphics[width=\linewidth]{media/output/features/compare/combined_it_2/cmp_global_features_TagVarCSV_vertexCategory.pdf}
    \caption*{Input similarity for PIP-PGD(2).}
  \end{subfigure}\hfill
  \begin{subfigure}[t]{0.32\textwidth}
    \includegraphics[width=\linewidth]{media/output/features/compare/combined_it_3/cmp_global_features_TagVarCSV_vertexCategory.pdf}
    \caption*{Input similarity for PIP-PGD(3).}
  \end{subfigure}

  \caption*{Histogram of \texttt{TagVarCSV\_vertexCategory} for multiple iterations of PIP-PGD tested against nominal inputs.}
  \label{fig:combined_input_TagVarCSV_vertexCategory}
\end{figure}

\newpage
\subsection*{CPF Features}


\begin{figure}[htbp]
  \centering
  \begin{subfigure}[t]{0.32\textwidth}
    \includegraphics[width=\linewidth]{media/output/features/compare/combined_it_1/cmp_cpf_arr_Cpfcan_chi2.pdf}
    \caption*{Input similarity for PIP-PGD(1).}
  \end{subfigure}\hfill
  \begin{subfigure}[t]{0.32\textwidth}
    \includegraphics[width=\linewidth]{media/output/features/compare/combined_it_2/cmp_cpf_arr_Cpfcan_chi2.pdf}
    \caption*{Input similarity for PIP-PGD(2).}
  \end{subfigure}\hfill
  \begin{subfigure}[t]{0.32\textwidth}
    \includegraphics[width=\linewidth]{media/output/features/compare/combined_it_3/cmp_cpf_arr_Cpfcan_chi2.pdf}
    \caption*{Input similarity for PIP-PGD(3).}
  \end{subfigure}

  \caption*{Histogram of \texttt{Cpfcan\_chi2} for multiple iterations of PIP-PGD tested against nominal inputs.}
  \label{fig:combined_input_Cpfcan_chi2}
\end{figure}

\begin{figure}[htbp]
  \centering
  \begin{subfigure}[t]{0.32\textwidth}
    \includegraphics[width=\linewidth]{media/output/features/compare/combined_it_1/cmp_cpf_arr_Cpfcan_chi2.pdf}
    \caption*{Input similarity for PIP-PGD(1).}
  \end{subfigure}\hfill
  \begin{subfigure}[t]{0.32\textwidth}
    \includegraphics[width=\linewidth]{media/output/features/compare/combined_it_2/cmp_cpf_arr_Cpfcan_chi2.pdf}
    \caption*{Input similarity for PIP-PGD(2).}
  \end{subfigure}\hfill
  \begin{subfigure}[t]{0.32\textwidth}
    \includegraphics[width=\linewidth]{media/output/features/compare/combined_it_3/cmp_cpf_arr_Cpfcan_chi2.pdf}
    \caption*{Input similarity for PIP-PGD(3).}
  \end{subfigure}

  \caption*{Histogram of \texttt{Cpfcan\_chi2} for multiple iterations of PIP-PGD tested against nominal inputs.}
  \label{fig:combined_input_Cpfcan_chi2}
\end{figure}

\begin{figure}[htbp]
  \centering
  \begin{subfigure}[t]{0.32\textwidth}
    \includegraphics[width=\linewidth]{media/output/features/compare/combined_it_1/cmp_cpf_arr_Cpfcan_drminsv.pdf}
    \caption*{Input similarity for PIP-PGD(1).}
  \end{subfigure}\hfill
  \begin{subfigure}[t]{0.32\textwidth}
    \includegraphics[width=\linewidth]{media/output/features/compare/combined_it_2/cmp_cpf_arr_Cpfcan_drminsv.pdf}
    \caption*{Input similarity for PIP-PGD(2).}
  \end{subfigure}\hfill
  \begin{subfigure}[t]{0.32\textwidth}
    \includegraphics[width=\linewidth]{media/output/features/compare/combined_it_3/cmp_cpf_arr_Cpfcan_drminsv.pdf}
    \caption*{Input similarity for PIP-PGD(3).}
  \end{subfigure}

  \caption*{Histogram of \texttt{Cpfcan\_drminsv} for multiple iterations of PIP-PGD tested against nominal inputs.}
  \label{fig:combined_input_Cpfcan_drminsv}
\end{figure}

\begin{figure}[htbp]
  \centering
  \begin{subfigure}[t]{0.32\textwidth}
    \includegraphics[width=\linewidth]{media/output/features/compare/combined_it_1/cmp_cpf_arr_Cpfcan_ptrel.pdf}
    \caption*{Input similarity for PIP-PGD(1).}
  \end{subfigure}\hfill
  \begin{subfigure}[t]{0.32\textwidth}
    \includegraphics[width=\linewidth]{media/output/features/compare/combined_it_2/cmp_cpf_arr_Cpfcan_ptrel.pdf}
    \caption*{Input similarity for PIP-PGD(2).}
  \end{subfigure}\hfill
  \begin{subfigure}[t]{0.32\textwidth}
    \includegraphics[width=\linewidth]{media/output/features/compare/combined_it_3/cmp_cpf_arr_Cpfcan_ptrel.pdf}
    \caption*{Input similarity for PIP-PGD(3).}
  \end{subfigure}

  \caption*{Histogram of \texttt{Cpfcan\_ptrel} for multiple iterations of PIP-PGD tested against nominal inputs.}
  \label{fig:combined_input_Cpfcan_ptrel}
\end{figure}

\begin{figure}[htbp]
  \centering
  \begin{subfigure}[t]{0.32\textwidth}
    \includegraphics[width=\linewidth]{media/output/features/compare/combined_it_1/cmp_cpf_arr_Cpfcan_puppiw.pdf}
    \caption*{Input similarity for PIP-PGD(1).}
  \end{subfigure}\hfill
  \begin{subfigure}[t]{0.32\textwidth}
    \includegraphics[width=\linewidth]{media/output/features/compare/combined_it_2/cmp_cpf_arr_Cpfcan_puppiw.pdf}
    \caption*{Input similarity for PIP-PGD(2).}
  \end{subfigure}\hfill
  \begin{subfigure}[t]{0.32\textwidth}
    \includegraphics[width=\linewidth]{media/output/features/compare/combined_it_3/cmp_cpf_arr_Cpfcan_puppiw.pdf}
    \caption*{Input similarity for PIP-PGD(3).}
  \end{subfigure}

  \caption*{Histogram of \texttt{Cpfcan\_puppiw} for multiple iterations of PIP-PGD tested against nominal inputs.}
  \label{fig:combined_input_Cpfcan_puppiw}
\end{figure}

\begin{figure}[htbp]
  \centering
  \begin{subfigure}[t]{0.32\textwidth}
    \includegraphics[width=\linewidth]{media/output/features/compare/combined_it_1/cmp_cpf_arr_Cpfcan_quality.pdf}
    \caption*{Input similarity for PIP-PGD(1).}
  \end{subfigure}\hfill
  \begin{subfigure}[t]{0.32\textwidth}
    \includegraphics[width=\linewidth]{media/output/features/compare/combined_it_2/cmp_cpf_arr_Cpfcan_quality.pdf}
    \caption*{Input similarity for PIP-PGD(2).}
  \end{subfigure}\hfill
  \begin{subfigure}[t]{0.32\textwidth}
    \includegraphics[width=\linewidth]{media/output/features/compare/combined_it_3/cmp_cpf_arr_Cpfcan_quality.pdf}
    \caption*{Input similarity for PIP-PGD(3).}
  \end{subfigure}

  \caption*{Histogram of \texttt{Cpfcan\_quality} for multiple iterations of PIP-PGD tested against nominal inputs.}
  \label{fig:combined_input_Cpfcan_quality}
\end{figure}

\begin{figure}[htbp]
  \centering
  \begin{subfigure}[t]{0.32\textwidth}
    \includegraphics[width=\linewidth]{media/output/features/compare/combined_it_1/cmp_cpf_arr_Cpfcan_VTX_ass.pdf}
    \caption*{Input similarity for PIP-PGD(1).}
  \end{subfigure}\hfill
  \begin{subfigure}[t]{0.32\textwidth}
    \includegraphics[width=\linewidth]{media/output/features/compare/combined_it_2/cmp_cpf_arr_Cpfcan_VTX_ass.pdf}
    \caption*{Input similarity for PIP-PGD(2).}
  \end{subfigure}\hfill
  \begin{subfigure}[t]{0.32\textwidth}
    \includegraphics[width=\linewidth]{media/output/features/compare/combined_it_3/cmp_cpf_arr_Cpfcan_VTX_ass.pdf}
    \caption*{Input similarity for PIP-PGD(3).}
  \end{subfigure}

  \caption*{Histogram of \texttt{Cpfcan\_VTX\_ass} for multiple iterations of PIP-PGD tested against nominal inputs.}
  \label{fig:combined_input_Cpfcan_VTX_ass}
\end{figure}

\begin{figure}[htbp]
  \centering
  \begin{subfigure}[t]{0.32\textwidth}
    \includegraphics[width=\linewidth]{media/output/features/compare/combined_it_1/cmp_cpf_arr_Cpfcan_BtagPf_trackDeltaR.pdf}
    \caption*{Input similarity for PIP-PGD(1).}
  \end{subfigure}\hfill
  \begin{subfigure}[t]{0.32\textwidth}
    \includegraphics[width=\linewidth]{media/output/features/compare/combined_it_2/cmp_cpf_arr_Cpfcan_BtagPf_trackDeltaR.pdf}
    \caption*{Input similarity for PIP-PGD(2).}
  \end{subfigure}\hfill
  \begin{subfigure}[t]{0.32\textwidth}
    \includegraphics[width=\linewidth]{media/output/features/compare/combined_it_3/cmp_cpf_arr_Cpfcan_BtagPf_trackDeltaR.pdf}
    \caption*{Input similarity for PIP-PGD(3).}
  \end{subfigure}

  \caption*{Histogram of \texttt{Cpfcan\_BtagPf\_trackDeltaR} for multiple iterations of PIP-PGD tested against nominal inputs.}
  \label{fig:combined_input_Cpfcan_BtagPf_trackDeltaR}
\end{figure}

\begin{figure}[htbp]
  \centering
  \begin{subfigure}[t]{0.32\textwidth}
    \includegraphics[width=\linewidth]{media/output/features/compare/combined_it_1/cmp_cpf_arr_Cpfcan_BtagPf_trackEtaRel.pdf}
    \caption*{Input similarity for PIP-PGD(1).}
  \end{subfigure}\hfill
  \begin{subfigure}[t]{0.32\textwidth}
    \includegraphics[width=\linewidth]{media/output/features/compare/combined_it_2/cmp_cpf_arr_Cpfcan_BtagPf_trackEtaRel.pdf}
    \caption*{Input similarity for PIP-PGD(2).}
  \end{subfigure}\hfill
  \begin{subfigure}[t]{0.32\textwidth}
    \includegraphics[width=\linewidth]{media/output/features/compare/combined_it_3/cmp_cpf_arr_Cpfcan_BtagPf_trackEtaRel.pdf}
    \caption*{Input similarity for PIP-PGD(3).}
  \end{subfigure}

  \caption*{Histogram of \texttt{Cpfcan\_BtagPf\_trackEtaRel} for multiple iterations of PIP-PGD tested against nominal inputs.}
  \label{fig:combined_input_Cpfcan_BtagPf_trackEtaRel}
\end{figure}

\begin{figure}[htbp]
  \centering
  \begin{subfigure}[t]{0.32\textwidth}
    \includegraphics[width=\linewidth]{media/output/features/compare/combined_it_1/cmp_cpf_arr_Cpfcan_BtagPf_trackJetDistVal.pdf}
    \caption*{Input similarity for PIP-PGD(1).}
  \end{subfigure}\hfill
  \begin{subfigure}[t]{0.32\textwidth}
    \includegraphics[width=\linewidth]{media/output/features/compare/combined_it_2/cmp_cpf_arr_Cpfcan_BtagPf_trackJetDistVal.pdf}
    \caption*{Input similarity for PIP-PGD(2).}
  \end{subfigure}\hfill
  \begin{subfigure}[t]{0.32\textwidth}
    \includegraphics[width=\linewidth]{media/output/features/compare/combined_it_3/cmp_cpf_arr_Cpfcan_BtagPf_trackJetDistVal.pdf}
    \caption*{Input similarity for PIP-PGD(3).}
  \end{subfigure}

  \caption*{Histogram of \texttt{Cpfcan\_BtagPf\_trackJetDistVal} for multiple iterations of PIP-PGD tested against nominal inputs.}
  \label{fig:combined_input_Cpfcan_BtagPf_trackJetDistVal}
\end{figure}

\begin{figure}[htbp]
  \centering
  \begin{subfigure}[t]{0.32\textwidth}
    \includegraphics[width=\linewidth]{media/output/features/compare/combined_it_1/cmp_cpf_arr_Cpfcan_BtagPf_trackPPar.pdf}
    \caption*{Input similarity for PIP-PGD(1).}
  \end{subfigure}\hfill
  \begin{subfigure}[t]{0.32\textwidth}
    \includegraphics[width=\linewidth]{media/output/features/compare/combined_it_2/cmp_cpf_arr_Cpfcan_BtagPf_trackPPar.pdf}
    \caption*{Input similarity for PIP-PGD(2).}
  \end{subfigure}\hfill
  \begin{subfigure}[t]{0.32\textwidth}
    \includegraphics[width=\linewidth]{media/output/features/compare/combined_it_3/cmp_cpf_arr_Cpfcan_BtagPf_trackPPar.pdf}
    \caption*{Input similarity for PIP-PGD(3).}
  \end{subfigure}

  \caption*{Histogram of \texttt{Cpfcan\_BtagPf\_trackPPar} for multiple iterations of PIP-PGD tested against nominal inputs.}
  \label{fig:combined_input_Cpfcan_BtagPf_trackPPar}
\end{figure}

\begin{figure}[htbp]
  \centering
  \begin{subfigure}[t]{0.32\textwidth}
    \includegraphics[width=\linewidth]{media/output/features/compare/combined_it_1/cmp_cpf_arr_Cpfcan_BtagPf_trackPParRatio.pdf}
    \caption*{Input similarity for PIP-PGD(1).}
  \end{subfigure}\hfill
  \begin{subfigure}[t]{0.32\textwidth}
    \includegraphics[width=\linewidth]{media/output/features/compare/combined_it_2/cmp_cpf_arr_Cpfcan_BtagPf_trackPParRatio.pdf}
    \caption*{Input similarity for PIP-PGD(2).}
  \end{subfigure}\hfill
  \begin{subfigure}[t]{0.32\textwidth}
    \includegraphics[width=\linewidth]{media/output/features/compare/combined_it_3/cmp_cpf_arr_Cpfcan_BtagPf_trackPParRatio.pdf}
    \caption*{Input similarity for PIP-PGD(3).}
  \end{subfigure}

  \caption*{Histogram of \texttt{Cpfcan\_BtagPf\_trackPParRatio} for multiple iterations of PIP-PGD tested against nominal inputs.}
  \label{fig:combined_input_Cpfcan_BtagPf_trackPParRatio}
\end{figure}

\begin{figure}[htbp]
  \centering
  \begin{subfigure}[t]{0.32\textwidth}
    \includegraphics[width=\linewidth]{media/output/features/compare/combined_it_1/cmp_cpf_arr_Cpfcan_BtagPf_trackPtRel.pdf}
    \caption*{Input similarity for PIP-PGD(1).}
  \end{subfigure}\hfill
  \begin{subfigure}[t]{0.32\textwidth}
    \includegraphics[width=\linewidth]{media/output/features/compare/combined_it_2/cmp_cpf_arr_Cpfcan_BtagPf_trackPtRel.pdf}
    \caption*{Input similarity for PIP-PGD(2).}
  \end{subfigure}\hfill
  \begin{subfigure}[t]{0.32\textwidth}
    \includegraphics[width=\linewidth]{media/output/features/compare/combined_it_3/cmp_cpf_arr_Cpfcan_BtagPf_trackPtRel.pdf}
    \caption*{Input similarity for PIP-PGD(3).}
  \end{subfigure}

  \caption*{Histogram of \texttt{Cpfcan\_BtagPf\_trackPtRel} for multiple iterations of PIP-PGD tested against nominal inputs.}
  \label{fig:combined_input_Cpfcan_BtagPf_trackPtRel}
\end{figure}

\begin{figure}[htbp]
  \centering
  \begin{subfigure}[t]{0.32\textwidth}
    \includegraphics[width=\linewidth]{media/output/features/compare/combined_it_1/cmp_cpf_arr_Cpfcan_BtagPf_trackSip2dSig.pdf}
    \caption*{Input similarity for PIP-PGD(1).}
  \end{subfigure}\hfill
  \begin{subfigure}[t]{0.32\textwidth}
    \includegraphics[width=\linewidth]{media/output/features/compare/combined_it_2/cmp_cpf_arr_Cpfcan_BtagPf_trackSip2dSig.pdf}
    \caption*{Input similarity for PIP-PGD(2).}
  \end{subfigure}\hfill
  \begin{subfigure}[t]{0.32\textwidth}
    \includegraphics[width=\linewidth]{media/output/features/compare/combined_it_3/cmp_cpf_arr_Cpfcan_BtagPf_trackSip2dSig.pdf}
    \caption*{Input similarity for PIP-PGD(3).}
  \end{subfigure}

  \caption*{Histogram of \texttt{Cpfcan\_BtagPf\_trackSip2dSig} for multiple iterations of PIP-PGD tested against nominal inputs.}
  \label{fig:combined_input_Cpfcan_BtagPf_trackSip2dSig}
\end{figure}

\begin{figure}[htbp]
  \centering
  \begin{subfigure}[t]{0.32\textwidth}
    \includegraphics[width=\linewidth]{media/output/features/compare/combined_it_1/cmp_cpf_arr_Cpfcan_BtagPf_trackSip2dVal.pdf}
    \caption*{Input similarity for PIP-PGD(1).}
  \end{subfigure}\hfill
  \begin{subfigure}[t]{0.32\textwidth}
    \includegraphics[width=\linewidth]{media/output/features/compare/combined_it_2/cmp_cpf_arr_Cpfcan_BtagPf_trackSip2dVal.pdf}
    \caption*{Input similarity for PIP-PGD(2).}
  \end{subfigure}\hfill
  \begin{subfigure}[t]{0.32\textwidth}
    \includegraphics[width=\linewidth]{media/output/features/compare/combined_it_3/cmp_cpf_arr_Cpfcan_BtagPf_trackSip2dVal.pdf}
    \caption*{Input similarity for PIP-PGD(3).}
  \end{subfigure}

  \caption*{Histogram of \texttt{Cpfcan\_BtagPf\_trackSip2dVal} for multiple iterations of PIP-PGD tested against nominal inputs.}
  \label{fig:combined_input_Cpfcan_BtagPf_trackSip2dVal}
\end{figure}

\begin{figure}[htbp]
  \centering
  \begin{subfigure}[t]{0.32\textwidth}
    \includegraphics[width=\linewidth]{media/output/features/compare/combined_it_1/cmp_cpf_arr_Cpfcan_BtagPf_trackSip3dSig.pdf}
    \caption*{Input similarity for PIP-PGD(1).}
  \end{subfigure}\hfill
  \begin{subfigure}[t]{0.32\textwidth}
    \includegraphics[width=\linewidth]{media/output/features/compare/combined_it_2/cmp_cpf_arr_Cpfcan_BtagPf_trackSip3dSig.pdf}
    \caption*{Input similarity for PIP-PGD(2).}
  \end{subfigure}\hfill
  \begin{subfigure}[t]{0.32\textwidth}
    \includegraphics[width=\linewidth]{media/output/features/compare/combined_it_3/cmp_cpf_arr_Cpfcan_BtagPf_trackSip3dSig.pdf}
    \caption*{Input similarity for PIP-PGD(3).}
  \end{subfigure}

  \caption*{Histogram of \texttt{Cpfcan\_BtagPf\_trackSip3dSig} for multiple iterations of PIP-PGD tested against nominal inputs.}
  \label{fig:combined_input_Cpfcan_BtagPf_trackSip3dSig}
\end{figure}

\begin{figure}[htbp]
  \centering
  \begin{subfigure}[t]{0.32\textwidth}
    \includegraphics[width=\linewidth]{media/output/features/compare/combined_it_1/cmp_cpf_arr_Cpfcan_BtagPf_trackSip3dVal.pdf}
    \caption*{Input similarity for PIP-PGD(1).}
  \end{subfigure}\hfill
  \begin{subfigure}[t]{0.32\textwidth}
    \includegraphics[width=\linewidth]{media/output/features/compare/combined_it_2/cmp_cpf_arr_Cpfcan_BtagPf_trackSip3dVal.pdf}
    \caption*{Input similarity for PIP-PGD(2).}
  \end{subfigure}\hfill
  \begin{subfigure}[t]{0.32\textwidth}
    \includegraphics[width=\linewidth]{media/output/features/compare/combined_it_3/cmp_cpf_arr_Cpfcan_BtagPf_trackSip3dVal.pdf}
    \caption*{Input similarity for PIP-PGD(3).}
  \end{subfigure}

  \caption*{Histogram of \texttt{Cpfcan\_BtagPf\_trackSip3dVal} for multiple iterations of PIP-PGD tested against nominal inputs.}
  \label{fig:combined_input_Cpfcan_BtagPf_trackSip3dVal}
\end{figure}

\newpage
\subsection*{NPF Features}


\begin{figure}[htbp]
  \centering
  \begin{subfigure}[t]{0.32\textwidth}
    \includegraphics[width=\linewidth]{media/output/features/compare/combined_it_1/cmp_npf_arr_Npfcan_deltaR.pdf}
    \caption*{Input similarity for PIP-PGD(1).}
  \end{subfigure}\hfill
  \begin{subfigure}[t]{0.32\textwidth}
    \includegraphics[width=\linewidth]{media/output/features/compare/combined_it_2/cmp_npf_arr_Npfcan_deltaR.pdf}
    \caption*{Input similarity for PIP-PGD(2).}
  \end{subfigure}\hfill
  \begin{subfigure}[t]{0.32\textwidth}
    \includegraphics[width=\linewidth]{media/output/features/compare/combined_it_3/cmp_npf_arr_Npfcan_deltaR.pdf}
    \caption*{Input similarity for PIP-PGD(3).}
  \end{subfigure}

  \caption*{Histogram of \texttt{Npfcan\_deltaR} for multiple iterations of PIP-PGD tested against nominal inputs.}
  \label{fig:combined_input_Npfcan_deltaR}
\end{figure}

\begin{figure}[htbp]
  \centering
  \begin{subfigure}[t]{0.32\textwidth}
    \includegraphics[width=\linewidth]{media/output/features/compare/combined_it_1/cmp_npf_arr_Npfcan_drminsv.pdf}
    \caption*{Input similarity for PIP-PGD(1).}
  \end{subfigure}\hfill
  \begin{subfigure}[t]{0.32\textwidth}
    \includegraphics[width=\linewidth]{media/output/features/compare/combined_it_2/cmp_npf_arr_Npfcan_drminsv.pdf}
    \caption*{Input similarity for PIP-PGD(2).}
  \end{subfigure}\hfill
  \begin{subfigure}[t]{0.32\textwidth}
    \includegraphics[width=\linewidth]{media/output/features/compare/combined_it_3/cmp_npf_arr_Npfcan_drminsv.pdf}
    \caption*{Input similarity for PIP-PGD(3).}
  \end{subfigure}

  \caption*{Histogram of \texttt{Npfcan\_drminsv} for multiple iterations of PIP-PGD tested against nominal inputs.}
  \label{fig:combined_input_Npfcan_drminsv}
\end{figure}

\begin{figure}[htbp]
  \centering
  \begin{subfigure}[t]{0.32\textwidth}
    \includegraphics[width=\linewidth]{media/output/features/compare/combined_it_1/cmp_npf_arr_Npfcan_HadFrac.pdf}
    \caption*{Input similarity for PIP-PGD(1).}
  \end{subfigure}\hfill
  \begin{subfigure}[t]{0.32\textwidth}
    \includegraphics[width=\linewidth]{media/output/features/compare/combined_it_2/cmp_npf_arr_Npfcan_HadFrac.pdf}
    \caption*{Input similarity for PIP-PGD(2).}
  \end{subfigure}\hfill
  \begin{subfigure}[t]{0.32\textwidth}
    \includegraphics[width=\linewidth]{media/output/features/compare/combined_it_3/cmp_npf_arr_Npfcan_HadFrac.pdf}
    \caption*{Input similarity for PIP-PGD(3).}
  \end{subfigure}

  \caption*{Histogram of \texttt{Npfcan\_HadFrac} for multiple iterations of PIP-PGD tested against nominal inputs.}
  \label{fig:combined_input_Npfcan_HadFrac}
\end{figure}

\begin{figure}[htbp]
  \centering
  \begin{subfigure}[t]{0.32\textwidth}
    \includegraphics[width=\linewidth]{media/output/features/compare/combined_it_1/cmp_npf_arr_Npfcan_isGamma.pdf}
    \caption*{Input similarity for PIP-PGD(1).}
  \end{subfigure}\hfill
  \begin{subfigure}[t]{0.32\textwidth}
    \includegraphics[width=\linewidth]{media/output/features/compare/combined_it_2/cmp_npf_arr_Npfcan_isGamma.pdf}
    \caption*{Input similarity for PIP-PGD(2).}
  \end{subfigure}\hfill
  \begin{subfigure}[t]{0.32\textwidth}
    \includegraphics[width=\linewidth]{media/output/features/compare/combined_it_3/cmp_npf_arr_Npfcan_isGamma.pdf}
    \caption*{Input similarity for PIP-PGD(3).}
  \end{subfigure}

  \caption*{Histogram of \texttt{Npfcan\_isGamma} for multiple iterations of PIP-PGD tested against nominal inputs.}
  \label{fig:combined_input_Npfcan_isGamma}
\end{figure}

\begin{figure}[htbp]
  \centering
  \begin{subfigure}[t]{0.32\textwidth}
    \includegraphics[width=\linewidth]{media/output/features/compare/combined_it_1/cmp_npf_arr_Npfcan_ptrel.pdf}
    \caption*{Input similarity for PIP-PGD(1).}
  \end{subfigure}\hfill
  \begin{subfigure}[t]{0.32\textwidth}
    \includegraphics[width=\linewidth]{media/output/features/compare/combined_it_2/cmp_npf_arr_Npfcan_ptrel.pdf}
    \caption*{Input similarity for PIP-PGD(2).}
  \end{subfigure}\hfill
  \begin{subfigure}[t]{0.32\textwidth}
    \includegraphics[width=\linewidth]{media/output/features/compare/combined_it_3/cmp_npf_arr_Npfcan_ptrel.pdf}
    \caption*{Input similarity for PIP-PGD(3).}
  \end{subfigure}

  \caption*{Histogram of \texttt{Npfcan\_ptrel} for multiple iterations of PIP-PGD tested against nominal inputs.}
  \label{fig:combined_input_Npfcan_ptrel}
\end{figure}

\begin{figure}[htbp]
  \centering
  \begin{subfigure}[t]{0.32\textwidth}
    \includegraphics[width=\linewidth]{media/output/features/compare/combined_it_1/cmp_npf_arr_Npfcan_puppiw.pdf}
    \caption*{Input similarity for PIP-PGD(1).}
  \end{subfigure}\hfill
  \begin{subfigure}[t]{0.32\textwidth}
    \includegraphics[width=\linewidth]{media/output/features/compare/combined_it_2/cmp_npf_arr_Npfcan_puppiw.pdf}
    \caption*{Input similarity for PIP-PGD(2).}
  \end{subfigure}\hfill
  \begin{subfigure}[t]{0.32\textwidth}
    \includegraphics[width=\linewidth]{media/output/features/compare/combined_it_3/cmp_npf_arr_Npfcan_puppiw.pdf}
    \caption*{Input similarity for PIP-PGD(3).}
  \end{subfigure}

  \caption*{Histogram of \texttt{Npfcan\_puppiw} for multiple iterations of PIP-PGD tested against nominal inputs.}
  \label{fig:combined_input_Npfcan_puppiw}
\end{figure}


\newpage
\subsection*{SV Features}

\begin{figure}[htbp]
  \centering
  \begin{subfigure}[t]{0.32\textwidth}
    \includegraphics[width=\linewidth]{media/output/features/compare/combined_it_1/cmp_vtx_arr_sv_chi2.pdf}
    \caption*{Input similarity for PIP-PGD(1).}
  \end{subfigure}\hfill
  \begin{subfigure}[t]{0.32\textwidth}
    \includegraphics[width=\linewidth]{media/output/features/compare/combined_it_2/cmp_vtx_arr_sv_chi2.pdf}
    \caption*{Input similarity for PIP-PGD(2).}
  \end{subfigure}\hfill
  \begin{subfigure}[t]{0.32\textwidth}
    \includegraphics[width=\linewidth]{media/output/features/compare/combined_it_3/cmp_vtx_arr_sv_chi2.pdf}
    \caption*{Input similarity for PIP-PGD(3).}
  \end{subfigure}

  \caption*{Histogram of \texttt{sv\_chi2} for multiple iterations of PIP-PGD tested against nominal inputs.}
  \label{fig:combined_input_sv_chi2}
\end{figure}

\begin{figure}[htbp]
  \centering
  \begin{subfigure}[t]{0.32\textwidth}
    \includegraphics[width=\linewidth]{media/output/features/compare/combined_it_1/cmp_vtx_arr_sv_costhetasvpv.pdf}
    \caption*{Input similarity for PIP-PGD(1).}
  \end{subfigure}\hfill
  \begin{subfigure}[t]{0.32\textwidth}
    \includegraphics[width=\linewidth]{media/output/features/compare/combined_it_2/cmp_vtx_arr_sv_costhetasvpv.pdf}
    \caption*{Input similarity for PIP-PGD(2).}
  \end{subfigure}\hfill
  \begin{subfigure}[t]{0.32\textwidth}
    \includegraphics[width=\linewidth]{media/output/features/compare/combined_it_3/cmp_vtx_arr_sv_costhetasvpv.pdf}
    \caption*{Input similarity for PIP-PGD(3).}
  \end{subfigure}

  \caption*{Histogram of \texttt{sv\_costhetasvpv} for multiple iterations of PIP-PGD tested against nominal inputs.}
  \label{fig:combined_input_sv_costhetasvpv}
\end{figure}

\begin{figure}[htbp]
  \centering
  \begin{subfigure}[t]{0.32\textwidth}
    \includegraphics[width=\linewidth]{media/output/features/compare/combined_it_1/cmp_vtx_arr_sv_d3d.pdf}
    \caption*{Input similarity for PIP-PGD(1).}
  \end{subfigure}\hfill
  \begin{subfigure}[t]{0.32\textwidth}
    \includegraphics[width=\linewidth]{media/output/features/compare/combined_it_2/cmp_vtx_arr_sv_d3d.pdf}
    \caption*{Input similarity for PIP-PGD(2).}
  \end{subfigure}\hfill
  \begin{subfigure}[t]{0.32\textwidth}
    \includegraphics[width=\linewidth]{media/output/features/compare/combined_it_3/cmp_vtx_arr_sv_d3d.pdf}
    \caption*{Input similarity for PIP-PGD(3).}
  \end{subfigure}

  \caption*{Histogram of \texttt{sv\_d3d} for multiple iterations of PIP-PGD tested against nominal inputs.}
  \label{fig:combined_input_sv_d3d}
\end{figure}

\begin{figure}[htbp]
  \centering
  \begin{subfigure}[t]{0.32\textwidth}
    \includegraphics[width=\linewidth]{media/output/features/compare/combined_it_1/cmp_vtx_arr_sv_d3dsig.pdf}
    \caption*{Input similarity for PIP-PGD(1).}
  \end{subfigure}\hfill
  \begin{subfigure}[t]{0.32\textwidth}
    \includegraphics[width=\linewidth]{media/output/features/compare/combined_it_2/cmp_vtx_arr_sv_d3dsig.pdf}
    \caption*{Input similarity for PIP-PGD(2).}
  \end{subfigure}\hfill
  \begin{subfigure}[t]{0.32\textwidth}
    \includegraphics[width=\linewidth]{media/output/features/compare/combined_it_3/cmp_vtx_arr_sv_d3dsig.pdf}
    \caption*{Input similarity for PIP-PGD(3).}
  \end{subfigure}

  \caption*{Histogram of \texttt{sv\_d3dsig} for multiple iterations of PIP-PGD tested against nominal inputs.}
  \label{fig:combined_input_sv_d3dsig}
\end{figure}

\begin{figure}[htbp]
  \centering
  \begin{subfigure}[t]{0.32\textwidth}
    \includegraphics[width=\linewidth]{media/output/features/compare/combined_it_1/cmp_vtx_arr_sv_deltaR.pdf}
    \caption*{Input similarity for PIP-PGD(1).}
  \end{subfigure}\hfill
  \begin{subfigure}[t]{0.32\textwidth}
    \includegraphics[width=\linewidth]{media/output/features/compare/combined_it_2/cmp_vtx_arr_sv_deltaR.pdf}
    \caption*{Input similarity for PIP-PGD(2).}
  \end{subfigure}\hfill
  \begin{subfigure}[t]{0.32\textwidth}
    \includegraphics[width=\linewidth]{media/output/features/compare/combined_it_3/cmp_vtx_arr_sv_deltaR.pdf}
    \caption*{Input similarity for PIP-PGD(3).}
  \end{subfigure}

  \caption*{Histogram of \texttt{sv\_deltaR} for multiple iterations of PIP-PGD tested against nominal inputs.}
  \label{fig:combined_input_sv_deltaR}
\end{figure}

\begin{figure}[htbp]
  \centering
  \begin{subfigure}[t]{0.32\textwidth}
    \includegraphics[width=\linewidth]{media/output/features/compare/combined_it_1/cmp_vtx_arr_sv_dxy.pdf}
    \caption*{Input similarity for PIP-PGD(1).}
  \end{subfigure}\hfill
  \begin{subfigure}[t]{0.32\textwidth}
    \includegraphics[width=\linewidth]{media/output/features/compare/combined_it_2/cmp_vtx_arr_sv_dxy.pdf}
    \caption*{Input similarity for PIP-PGD(2).}
  \end{subfigure}\hfill
  \begin{subfigure}[t]{0.32\textwidth}
    \includegraphics[width=\linewidth]{media/output/features/compare/combined_it_3/cmp_vtx_arr_sv_dxy.pdf}
    \caption*{Input similarity for PIP-PGD(3).}
  \end{subfigure}

  \caption*{Histogram of \texttt{sv\_dxy} for multiple iterations of PIP-PGD tested against nominal inputs.}
  \label{fig:combined_input_sv_dxy}
\end{figure}

\begin{figure}[htbp]
  \centering
  \begin{subfigure}[t]{0.32\textwidth}
    \includegraphics[width=\linewidth]{media/output/features/compare/combined_it_1/cmp_vtx_arr_sv_dxysig.pdf}
    \caption*{Input similarity for PIP-PGD(1).}
  \end{subfigure}\hfill
  \begin{subfigure}[t]{0.32\textwidth}
    \includegraphics[width=\linewidth]{media/output/features/compare/combined_it_2/cmp_vtx_arr_sv_dxysig.pdf}
    \caption*{Input similarity for PIP-PGD(2).}
  \end{subfigure}\hfill
  \begin{subfigure}[t]{0.32\textwidth}
    \includegraphics[width=\linewidth]{media/output/features/compare/combined_it_3/cmp_vtx_arr_sv_dxysig.pdf}
    \caption*{Input similarity for PIP-PGD(3).}
  \end{subfigure}

  \caption*{Histogram of \texttt{sv\_dxysig} for multiple iterations of PIP-PGD tested against nominal inputs.}
  \label{fig:combined_input_sv_dxysig}
\end{figure}

\begin{figure}[htbp]
  \centering
  \begin{subfigure}[t]{0.32\textwidth}
    \includegraphics[width=\linewidth]{media/output/features/compare/combined_it_1/cmp_vtx_arr_sv_enratio.pdf}
    \caption*{Input similarity for PIP-PGD(1).}
  \end{subfigure}\hfill
  \begin{subfigure}[t]{0.32\textwidth}
    \includegraphics[width=\linewidth]{media/output/features/compare/combined_it_2/cmp_vtx_arr_sv_enratio.pdf}
    \caption*{Input similarity for PIP-PGD(2).}
  \end{subfigure}\hfill
  \begin{subfigure}[t]{0.32\textwidth}
    \includegraphics[width=\linewidth]{media/output/features/compare/combined_it_3/cmp_vtx_arr_sv_enratio.pdf}
    \caption*{Input similarity for PIP-PGD(3).}
  \end{subfigure}

  \caption*{Histogram of \texttt{sv\_enratio} for multiple iterations of PIP-PGD tested against nominal inputs.}
  \label{fig:combined_input_sv_enratio}
\end{figure}

\begin{figure}[htbp]
  \centering
  \begin{subfigure}[t]{0.32\textwidth}
    \includegraphics[width=\linewidth]{media/output/features/compare/combined_it_1/cmp_vtx_arr_sv_mass.pdf}
    \caption*{Input similarity for PIP-PGD(1).}
  \end{subfigure}\hfill
  \begin{subfigure}[t]{0.32\textwidth}
    \includegraphics[width=\linewidth]{media/output/features/compare/combined_it_2/cmp_vtx_arr_sv_mass.pdf}
    \caption*{Input similarity for PIP-PGD(2).}
  \end{subfigure}\hfill
  \begin{subfigure}[t]{0.32\textwidth}
    \includegraphics[width=\linewidth]{media/output/features/compare/combined_it_3/cmp_vtx_arr_sv_mass.pdf}
    \caption*{Input similarity for PIP-PGD(3).}
  \end{subfigure}

  \caption*{Histogram of \texttt{sv\_mass} for multiple iterations of PIP-PGD tested against nominal inputs.}
  \label{fig:combined_input_sv_mass}
\end{figure}

\begin{figure}[htbp]
  \centering
  \begin{subfigure}[t]{0.32\textwidth}
    \includegraphics[width=\linewidth]{media/output/features/compare/combined_it_1/cmp_vtx_arr_sv_normchi2.pdf}
    \caption*{Input similarity for PIP-PGD(1).}
  \end{subfigure}\hfill
  \begin{subfigure}[t]{0.32\textwidth}
    \includegraphics[width=\linewidth]{media/output/features/compare/combined_it_2/cmp_vtx_arr_sv_normchi2.pdf}
    \caption*{Input similarity for PIP-PGD(2).}
  \end{subfigure}\hfill
  \begin{subfigure}[t]{0.32\textwidth}
    \includegraphics[width=\linewidth]{media/output/features/compare/combined_it_3/cmp_vtx_arr_sv_normchi2.pdf}
    \caption*{Input similarity for PIP-PGD(3).}
  \end{subfigure}

  \caption*{Histogram of \texttt{sv\_normchi2} for multiple iterations of PIP-PGD tested against nominal inputs.}
  \label{fig:combined_input_sv_normchi2}
\end{figure}

\begin{figure}[htbp]
  \centering
  \begin{subfigure}[t]{0.32\textwidth}
    \includegraphics[width=\linewidth]{media/output/features/compare/combined_it_1/cmp_vtx_arr_sv_ntracks.pdf}
    \caption*{Input similarity for PIP-PGD(1).}
  \end{subfigure}\hfill
  \begin{subfigure}[t]{0.32\textwidth}
    \includegraphics[width=\linewidth]{media/output/features/compare/combined_it_2/cmp_vtx_arr_sv_ntracks.pdf}
    \caption*{Input similarity for PIP-PGD(2).}
  \end{subfigure}\hfill
  \begin{subfigure}[t]{0.32\textwidth}
    \includegraphics[width=\linewidth]{media/output/features/compare/combined_it_3/cmp_vtx_arr_sv_ntracks.pdf}
    \caption*{Input similarity for PIP-PGD(3).}
  \end{subfigure}

  \caption*{Histogram of \texttt{sv\_ntracks} for multiple iterations of PIP-PGD tested against nominal inputs.}
  \label{fig:combined_input_sv_ntracks}
\end{figure}

\begin{figure}[htbp]
  \centering
  \begin{subfigure}[t]{0.32\textwidth}
    \includegraphics[width=\linewidth]{media/output/features/compare/combined_it_1/cmp_vtx_arr_sv_pt.pdf}
    \caption*{Input similarity for PIP-PGD(1).}
  \end{subfigure}\hfill
  \begin{subfigure}[t]{0.32\textwidth}
    \includegraphics[width=\linewidth]{media/output/features/compare/combined_it_2/cmp_vtx_arr_sv_pt.pdf}
    \caption*{Input similarity for PIP-PGD(2).}
  \end{subfigure}\hfill
  \begin{subfigure}[t]{0.32\textwidth}
    \includegraphics[width=\linewidth]{media/output/features/compare/combined_it_3/cmp_vtx_arr_sv_pt.pdf}
    \caption*{Input similarity for PIP-PGD(3).}
  \end{subfigure}

  \caption*{Histogram of \texttt{sv\_pt} for multiple iterations of PIP-PGD tested against nominal inputs.}
  \label{fig:combined_input_sv_pt}
\end{figure}

\newpage


\end{document}
